\documentclass[letter,11pt]{article}

\usepackage{amsfonts}
\usepackage{amsmath}
\usepackage{amssymb}
\usepackage[brazilian]{babel}
\usepackage{enumerate}
\usepackage[T1]{fontenc}
%\usepackage[ansinew,latin1]{inputenc}
\usepackage[utf8x]{inputenc}
\usepackage{multicol}
\setlength\columnseprule{0.5pt}

\newtheorem{exer}{Exercício}

\newcommand{\var}{Var}
\newcommand{\E}{\mathbb{E}}

\newcommand{\mat}[1]{\mbox{\boldmath{$#1$}}}

\usepackage[letterpaper,top=3cm, bottom=2cm, left=2.5cm, right=2.5cm]{geometry}

\begin{document}

%\thispagestyle{empty}
\begin{center}{ \Large MAT02023 - Inferência A }\end{center}

\begin{center}
{\large  \sc Lista 2 - Continuação de Conceitos de Probabilidade}
\end{center}
\vspace{15mm}

%%%%%%%%%%%%%%%%%%%%%%%%%%%%%%%%%%%%%%%%%%%%%%%%%%%%%%%%%%%%%%%%%%%%%%%%%%%%%%%

\noindent \textit{\textbf{Amostra Aleatória}}


\begin{exer} \rm
Um dado é lançado $n$ vezes, independentemente. Seja $X_1,\cdots,X_n$ os 
sucessivos valores das faces. É razoável se pensar que $X_1,\cdots,X_n$ é uma 
a.a. de uma v.a. $X$.
\begin{enumerate}[\bf(a)]
  \item Qual o modelo probabilístico para a v.a. $X$?
  \item Qual o modelo estatístico para o experimento?
  \item Mostre que $\E X = \mu_1 = 3,5$, $\E X^2 = \mu_2 = 91/6$ e $\E X^4 = 
  \mu_4 = 2275/6$.
  \item Obtenha uma distribuição aproximada para $\overline{X}$ e $M_2 = 
  \frac{1}{n}\sum_{i=1}^{n}X_i^2$.
  \item Assuma $n = 12$ e calcule aproximadamente $P(40 \leq \sum_{i=1}^{n}X_i 
  \leq 45)$ e $P(170 \leq \sum_{i=1}^{n}X_i^2\leq 190)$.
  \item Seja $N_i$ = ``num. de vezes em que ocorre a face $i$''. Qual a 
  distribuição de $N_i$? Obtenha uma distribuição aproximada para $N_i$.
\end{enumerate}
\end{exer}

% \medskip
\begin{exer} \rm
Bolas são sorteadas com reposição de uma urna contendo 1 bola branca e 2 bolas 
pretas. Denote $X_i = 0$ se a bola retirada no i-ésimo sorteio for branca e 
$X_i = 1$ se for preta. Considere a amostra aleatória $X_1, \ldots, X_9$:
\begin{enumerate}[a)]
  \item Qual a distribuição conjunta destas nove variáveis aleatórias?
  \item Qual a distribuição da soma destas variáveis?
  \item Encontre o valor esperado da média amostral.
  \item Encontre o valor esperado da variância amostral $S^2$.
\end{enumerate}
\end{exer}


% \medskip
\begin{exer} \rm
Seja $X_1, X_2, \ldots, X_n$ uma amostra aleatória de uma população 
Exponencial($\lambda$). Especificamente, $X_i$ poderia representar o tempo até 
a falha (medida em anos) para $n$ equipamentos idênticos colocados em teste.
\begin{enumerate}[a)]
  \item Encontre a distribuição conjunta das variáveis nesta amostra aleatória.
  \item Qual é a probabilidade de que todos os equipamentos durem mais de 2 anos?
\end{enumerate}
\end{exer}


% \medskip
\begin{exer} \rm
Seja $X_1,\cdots,X_n$  uma a.a. de uma população $X$ com distribuição Bernoulli 
com parâmetro $p$.
\begin{enumerate}[\bf(a)]
  \item Qual a distribuição conjunta da a.a.?
  \item Qual a distribuição de $X_1 +\cdots+ X_n$?
  \item Qual a distribuição de $\overline{X}$?
  \item Para $n = 2$, qual a distribuição de $\sum_{i}^{n}(X_i - p)^2$?
\end{enumerate}
\end{exer}


% \medskip
\begin{exer} \rm
Seja $X_1, X_2, \ldots, X_n$ uma a.a. tal que $X_1 \sim f_{\boldsymbol{\theta}}$. 
Determine a distribuição amostral de $\overline{X}$, quando
\begin{enumerate}[\bf(a)]
  \item $X_1 \sim \,\mbox{Poisson}(\lambda)$;
  \item $X_1 \sim \,\mbox{Exponencial}(\lambda)$.
\end{enumerate}
\end{exer}


\begin{exer}
\item Calcule a variância e esperança dos seguintes estimadores:
\begin{enumerate}[a)]
  \item $\overline{X}$.
  \item $S^2$.
\end{enumerate}
\end{exer}

\medskip
\medskip
\noindent \textit{\textbf{Função Geradora de Momentos}}

\medskip 
\begin{exer} \rm
Encontre a função geradora de momentos de uma variável aleatória $X \sim Poisson(\lambda)$.
\end{exer}


\begin{exer} \rm
Encontre a função geradora de momentos de uma variável aleatória $X \sim Exp(\lambda)$.
\end{exer}


\begin{exer} \rm
Seja $X$ uma variável aleatória com distribuição de probabilidade $Gama (\alpha, \beta)$. Determine a função geradora de momentos de X.
\end{exer}


\begin{exer} \rm
Suponha que X tem distribuição $\chi^2_n$. Mostre que $E(X) = n$ e $Var(X) = 2n$.
Sugestão: Use a função geradora de momentos.
\end{exer}


\begin{exer} \rm
Considere que $X$ e $Y$ são variáveis aleatórias independentes com distribuição 
normal padrão. Encontre a função geradora de momentos conjunta de $X$ e $Y$.
\end{exer}


\medskip
\medskip
\noindent \textit{\textbf{Teoremas Limite}}

\begin{exer} \rm
Suponha que uma população tem $\sigma=2$ e $\overline{X}$ é a média de amostras 
de tamanho 100. Encontre $l$ tal que $P(-l < \overline{X}-\mu < l)=0.9$.
\end{exer}


\begin{exer} \rm
Um pesquisador deseja estimar a média de uma população usando uma amostra grande 
o suficiente, tal que temos probabilidade de 0.95 que a média amostral não 
difira da média populacional por mais de 25\% do desvio padrão. Qual deve ser o 
tamanho da amostra?
\end{exer}


\begin{exer} \rm
Seja $\overline{X}$ a média de uma amostra aleatória de tamanho 75 com a 
seguinte função densidade $f(x) = I(0;1)(x)$. Calcule um valor aproximado para 
a seguinte probabilidade $P(0.45 < \overline{X} < 0.55)$.
\end{exer}


\begin{exer} \rm
Considere $X_1, X_2, \ldots, X_n$ uma amostra aleatória da distribuição 
Bernoulli$(p)$. Defina $Y= \sum_{i=1}^n X_i$.
\begin{enumerate}[a)]
  \item Qual é a distribuição de $Y$.
  \item Seja $n = 100$ e $p = 0.5$, utilize o Teorema Central do Limite para 
  calcular uma aproximação para a seguinte probabilidade $P(47,5 < Y < 52,5)$.
\end{enumerate}
\end{exer}


\begin{exer} \rm
Seja $Y \sim Binomial (400, 1/5)$, calcule um valor aproximado para a 
probabilidade $P(Y/n > 0.25)$.
\end{exer}


\begin{exer} \rm
Seja $f(x) = \frac{1}{x^2} I_{(1, \infty)}(x)$ a densidade de uma variável 
aleatória $X$. Considere uma amostra aleatória de tamanho 72 de uma população 
ue segue esta distribuição. Calcule, aproximadamente, a probabilidade de que 
mais de 50 observações da amostra aleatória sejam menores que 3.
\end{exer}


\begin{exer} \rm
Prove os seguintes teoremas e corolários:
\begin{enumerate}[a)]
  \item Teorema: Sejam $X_1, X_2, \ldots$ variáveis aleatórias i.i.d. com 
  $E(X_i)=\mu$ e $Var(X_i)=\sigma^2 < \infty$. 
  \vspace{0.4cm}
  Então para todo $\epsilon > 0$ temos que
  $$
  \lim_{n\rightarrow \infty}P(|\overline{X}_n-\mu| < \epsilon)=1.
  $$
  \vspace{0.4cm}
  Isto é, $\overline{X}_n$ converge em probabilidade para $\mu$.

  \item Teorema: Seja $f(\ldots)$ uma densidade com média $\mu$ e variância 
  finita $\sigma^2$. Considere $\overline{X}_n$ a média amostral de uma amostra 
  aleatória com tamanho $n$ obtida a partir de $f(\cdot)$. Seja $Z_n$ a seguinte 
  variável aleatória
  $$
  Z_n=\frac{\overline{X}_n-E(\overline{X}_n)}{\sqrt{Var(\overline{X}_n)}}
  $$
  A distribuição de $Z_n$ se aproxima da $N(0,1)$ conforme $n$ tende ao infinito.
  
  \item Teorema: Seja $X_1, X_2, \ldots, X_n$ uma amostra aleatória de uma 
  população com densidade $f(\cdot)$ com média $\mu$ e variância $\sigma^2$ então
  $$
  E(\overline{X})=\mu \mbox{ e } Var(\overline{X})=\frac{1}{n}\sigma^2
  $$
  
  \item Teorema: Se $X_1, X_2, \ldots, X_n$ são variáveis aleatórias Normais 
  independentes com médias $\mu_i$ e variâncias $\sigma_i^2$. Então
  $$
  U= \sum_{i=1}^k\left( \frac{X_i - \mu_i}{\sigma_i}\right)^2
  $$
  possui distribuição $\chi_k^2$.

  \item Corolário: Se $X_1, X_2, \ldots, X_n$ é uma amostra aleatória da 
  distribuição Normal com média $\mu$ e variância $\sigma^2$ então 
  $U=\sum_{i=1}^n \frac{(X_i-\mu)^2}{\sigma^2}$ tem distribuição $\chi^2_n$.

  \item Teorema: Se $Z_1, Z_2, \ldots, Z_n$ é uma amostra aleatória da 
  distribuição $N(0,1)$, então:
  \begin{enumerate}[i)]
    \item $\overline{Z} \sim N(0, 1/n)$
    \item $\overline{Z}$ e $\sum_{i=1}^n (Z_i-\overline{Z})^2$ são independentes
    \item $\sum_{i=1}^n (Z_i-\overline{Z})^2 \sim \chi^2_{n-1}$
  \end{enumerate}

  \item Teorema: Se $Z \sim N(0,1)$ e $U\sim \chi^2_k$ são variáveis aleatórias 
  independentes, então
  $$
  \frac{Z}{\sqrt{U/k}}\sim t_k.
  $$

\end{enumerate}
\end{exer}


\end{document}
