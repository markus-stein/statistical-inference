\documentclass[letter,11pt]{article}

\usepackage{amsfonts}
\usepackage{amsmath}
\usepackage{amssymb}
\usepackage[brazilian]{babel}
\usepackage{enumerate}
\usepackage[T1]{fontenc}
%\usepackage[ansinew,latin1]{inputenc}
\usepackage[utf8x]{inputenc}
\usepackage{multicol}
\usepackage{graphicx}
\setlength\columnseprule{0.5pt}

\newtheorem{exer}{Exercício}

\newcommand{\var}{Var}
\newcommand{\E}{\mathbb{E}}

\newcommand{\mat}[1]{\mbox{\boldmath{$#1$}}}

\usepackage[letterpaper,top=3cm, bottom=2cm, left=2.5cm, right=2.5cm]{geometry}

\begin{document}

%\thispagestyle{empty}
\begin{center}{ \Large MAT02023 - Inferência A }\end{center}

\begin{center}
{\large  \sc Gabarito Lista 5 - Família Exponencial e Suficiência}
\end{center}
\vspace{15mm}

%%%%%%%%%%%%%%%%%%%%%%%%%%%%%%%%%%%%%%%%%%%%%%%%%%%%%%%%%%%%%%%%%%%%%%%%%%%%%%%
% questoes 1 a 5
\includegraphics[scale=0.32]{gabarito_lista7_1Marcia.jpg}



\medskip
\setcounter{exer}{5}
% questao 6
\begin{exer} \rm
A função densidade de probabilidade conjunta de $\boldsymbol{X}$ é dada por
$$f(\boldsymbol{x}; \lambda) = \prod_{i=1}^n f(x_i; \lambda) \: I(x_i > 0) = \lambda^n \: e^{-\lambda \sum_{i=1}^n x_i} \: I(\boldsymbol{x} > \boldsymbol{0}).$$
E, pelo Teorema da Fatoração de Fisher-Neyman, podemos verificar que 
$$f(\boldsymbol{x}; \lambda) = g[\lambda; T(\boldsymbol{x})] \: h(\boldsymbol{x})$$
em que $g[\lambda; T(\boldsymbol{x})] = \lambda^n \: e^{-\lambda \sum_{i=1}^n x_i}$ e $h(\boldsymbol{x}) = I(\boldsymbol{x} > \boldsymbol{0})$. Então $T(\boldsymbol{x}) = \sum_{i=1}^n x_i$ é uma estatística suficiente para $\lambda$.

\end{exer}

% questao 7
\medskip
\begin{exer} \rm
\end{exer}

% questao 8
\medskip
\begin{exer} \rm
\end{exer}

% questao 9
\medskip
\begin{exer} \rm
\end{exer}

% questao 10
\medskip
\begin{exer} \rm
\begin{enumerate}[a)]
  \item Feita em aula.  
  \item Passos comentados em aula.  
\end{enumerate}
\end{exer}

% questao 11
\medskip
\begin{exer} \rm
\end{exer}
\includegraphics[scale=0.2]{gabarito_lista7_21Marcia.jpg}

% questao 12
\medskip
\begin{exer} \rm  
\end{exer}
\includegraphics[scale=0.2]{gabarito_lista7_22Marcia.jpg}


% questao 13
\medskip
\begin{exer} \rm  
\end{exer}

% questao 14
\medskip
\begin{exer} \rm
\begin{enumerate}[1)]
  \item Colocar na forma da família exponencial UNIDIMENSIONAL para $\boldsymbol{X} = \boldsymbol{x}$, por exemplo:
\begin{eqnarray}
f(\boldsymbol{x}; \eta}) & = & h(\boldsymbol{x}) \: c(\theta) \: exp\left[ \sum_{i=1}^{n} w(\theta) \: t(x_i) \right] \nonumber
\end{eqnarray}
em que $\theta \in \Theta$;

ou na parametrização natural
\begin{eqnarray}
f(\boldsymbol{x}; \eta}) & = & h(\boldsymbol{x}) \: b(\eta) \: exp\left[ \sum_{i=1}^{n} \eta \: t(x_i) \right] \nonumber
\end{eqnarray}
em que $\eta \in \Omega$;

  \item Verificar: $T(\boldsymbol{x}) = \sum_{i=1}^n t(x_i)$ é suficiente e completa para
  \begin{enumerate}[a)]
    \item  $\eta$ se $\Omega$ é um subconjunto dos reais $\mathbb{R}$;
    \item ou para $\theta$ se $\eta$ é uma função um a um de $\theta$ e $\Theta$ é um subconjunto dos reais $\mathbb{R}$.  
  \end{enumerate}
\end{enumerate}

\end{exer}

% questao 15
\medskip
\begin{exer} \rm
O mesmo que na questão 14 mas agora a família Exponencial é BIDIMENSIONAL, $\boldsymbol{\theta}$ ou $\boldsymbol{\eta}$.
Também, agora a estatística será do tipo $ \boldsymbol{T}($\boldsymbol{x}$) = (\sum_{i=1}^n t_1(x_i), \sum_{i=1}^n t_2(x_i))$
\end{exer}

\end{document}
