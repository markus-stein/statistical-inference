% ----------------------------------------------------------------
% AMS-LaTeX Paper ************************************************
% **** ---------------------------------------------------------
\documentclass[10pt,brazil,addpoints]{exam}
\usepackage{geometry}
\geometry{verbose,tmargin=1.5cm,bmargin=1.2cm,lmargin=1cm,rmargin=1.5cm}
\usepackage[T1]{fontenc}
\usepackage[utf8]{inputenc}
\usepackage{babel}
\usepackage{amsmath}
\usepackage{amsfonts}
\usepackage{mathtools}
\usepackage{breqn}
\usepackage[inline]{enumitem}

\usepackage[usenames,dvipsnames]{pstricks}
%\usepackage{epsfig}
\usepackage{pst-grad} % For gradients
\usepackage{pst-plot} % For axes

\pagestyle{plain}

% Para redefinir o texto ao lado das pontuações:
\pointpoints{ponto}{pontos}
% Para redefinir as margens:
%\extrawidth{1in}
%\extraheadheight{-.5in}

\usepackage{graphicx}

\usepackage{booktabs}

\usepackage{caption}

\usepackage{enumerate}

\setlength{\parindent}{0pt}

\usepackage{mleftright} % corrects spacing after \left( and before \right), etc
\mleftright

\newcommand{\E}{\mathbb{E}}
\newcommand{\Prob}{\mathbb{P}}

\begin{document}
% Título do Artigo
\title{GABARITO Lista 1}

% Definição do(s) autor(es). Observe que é possível colocar
% uma nota de agradecimento, ou algo semelhante.

\author{
  Profª. Márcia Barbian \\
  Disciplina: Inferência B\\
  %\and
  %Author 4 and Author 5  \\
  %Institution B
 % \and
  %Author 5  \\
  %Institution C
  \date{}
}

% Data. Se você quiser, pode entrar com o texto desejado no campo \date
% Caso queira a data do dia da compilação, exclua o comando \date
% Caso não queira que nada seja impresso no lugar da data, use \date{}

\maketitle


\begin{enumerate}[1.]

\item 

\begin{enumerate}[a)]
\item $Y\sim Exp(\theta)=Gama(1,\theta)$.
\item $X^\theta \sim U(0,1)$
\item $[\frac{\log(0.975)}{\log X} ; \frac{\log(0,025)}{\log X}]$.
\end{enumerate}

\item $[\overline{X}-1,64\sqrt{1/n};  \overline{X}+1,64\sqrt{1/n}]$


\item 
\begin{enumerate}[a)]
\item $\lambda\sum_{i=1}^{10} X_i \sim Gama(10,1)$.

\item $[\frac{5,425}{\sum_{i=1}^{10} X_i} ; \frac{15,705}{\sum_{i=1}^{10} X_i}]$

\item $2\lambda\sum_{i=1}^{10} X_i \sim Gama(10,1/2)$.
\end{enumerate}


\item 
\begin{enumerate}[a)]
\item $\beta(p)$ inclui as seguintes expressões:
$$
\sum_{y=7}^{20}  {20 \choose y}p^y(1-p)^{20-y} \qquad e \qquad \sum_{y=0}^1  {20 \choose y}p^y(1-p)^{20-y}
$$


\item $\beta(p) = \{1, 0.39, 0.16, 0.4, 0.75, 0.94, 0.99, 1, 1, 1, 1\}$. 
 
\end{enumerate}

\item 
\begin{enumerate}[a)]
 \item $\beta(p)$ inclui a expressão:
$$
\sum_{x=6}^{10}  {10 \choose x}\theta^x(1-\theta)^{10-x} 
$$
 
 \item $\beta(p) = \{0, 0, 0.01, 0.05, 0.17, 0.38, 0.63, 0.85, 0.97, 0.99, 1\}$. 
 
 \item 0,38
\end{enumerate}

\item 

\begin{enumerate}[a)]
 \item $\beta(\theta)=1-\frac{1}{2^\theta}$
 \item 0.5
\end{enumerate}



\end{enumerate}




\end{document}
% ----------------------------------------------------------------
