% ----------------------------------------------------------------
% AMS-LaTeX Paper ************************************************
% **** ---------------------------------------------------------
\documentclass[11pt,brazil,addpoints]{exam}
\usepackage{geometry}
\geometry{verbose,tmargin=1.5cm,bmargin=1.2cm,lmargin=1cm,rmargin=1.5cm}
\usepackage[T1]{fontenc}
\usepackage[utf8]{inputenc}
\usepackage{babel}
\usepackage{amsmath}
\usepackage{amsfonts}
\usepackage{mathtools}
\usepackage[usenames,dvipsnames]{pstricks}
\usepackage{amssymb}
\usepackage{amsthm}

% Para redefinir o texto ao lado das pontuações:
\pointpoints{ponto}{pontos}


\usepackage{graphicx}

\usepackage{booktabs}

\usepackage{caption}

\usepackage{enumerate}

\setlength{\parindent}{0pt}

\usepackage{mleftright} % corrects spacing after \left( and before \right), etc
\mleftright

\newcommand{\E}{\mathbb{E}}
\newcommand{\Prob}{\mathbb{P}}
\newcommand{\T}{\mathbb{T}}
\newcommand{\prob}{\mathbb{P}}

\begin{document}
% Título do Artigo
\title{Lista 2}

% Definição do(s) autor(es). Observe que é possível colocar
% uma nota de agradecimento, ou algo semelhante.

\author{
  Prof. Marcio Valk \\
  Disciplina: Inferência B\\
%
  \date{}
}

% Data. Se você quiser, pode entrar com o texto desejado no campo \date
% Caso queira a data do dia da compilação, exclua o comando \date
% Caso não queira que nada seja impresso no lugar da data, use \date{}

\maketitle


\begin{enumerate}[1.]

\item Na construção de intervalos de confiança para a proporção $p$ (ou parâmetro $p$ de uma distribuição Bernoulli) podemos utilizar três possíveis métodos como descrito na página 17 da apostila. Use o R para gerar amostras (n=10 e depois n=100) Bernoulli para um determinado parâmetro p (digamos p=0.2) e construa os Intervalos de Confiança com coeficiente de confiança $(1-\alpha)$100\% utilizando os três métodos. Sabe-se que ao repetir o experimento é esperado que os intervalos contenham o parâmetro em $(1-\alpha)$100\% das vezes. Então fixe $\alpha =0.05$, repita o experimento 1000 vezes para os três métodos e verifique se em 95\% das vezes o intervalo contém o parâmetro $p=0.2$. Qual sua conclusão? Os intervalos possuem mesmo a confiança desejada?


\end{enumerate}



\end{document}
% ----------------------------------------------------------------
