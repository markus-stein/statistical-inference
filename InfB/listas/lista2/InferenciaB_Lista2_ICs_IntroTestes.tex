\documentclass[letter,11pt]{article}

\usepackage{amsfonts}
\usepackage{amsmath}
\usepackage{amssymb}
\usepackage[brazilian]{babel}
\usepackage{enumerate}
\usepackage[T1]{fontenc}
%\usepackage[ansinew,latin1]{inputenc}
\usepackage[utf8x]{inputenc}
\usepackage{multicol}
\setlength\columnseprule{0.5pt}

\newtheorem{exer}{Exercício}
\newtheorem{teo}{Teorema}

\newcommand{\var}{Var}
\newcommand{\E}{\mathbb{E}}

\newcommand{\mat}[1]{\mbox{\boldmath{$#1$}}}

\usepackage[letterpaper,top=3cm, bottom=2cm, left=2.5cm, right=2.5cm]{geometry}

\begin{document}

%\thispagestyle{empty}
\begin{center}{ \Large MAT02026 - Inferência B }\end{center}

\begin{center}
{\large  \sc Lista 2 - ICs e Introdução aos Testes de Hipóteses}
\end{center}
\vspace{5mm}

%%%%%%%%%%%%%%%%%%%%%%%%%%%%%%%%%%%%%%%%%%%%%%%%%%%%%%%%%%%%%%%%%%%%%%%%%%%%%%%
\begin{exer} \rm
O que é uma quantidade pivotal? Exemplifique.
\end{exer}


\begin{exer} \rm
A afirmação: ``Há 95\% de probabilidade do parâmetro $\theta$ estar contido no intervalo $[0,3 ; 0,35]$" está correta? Justifique.
\end{exer}

\begin{exer} \rm
Seja $X_1, \ldots, X_n \sim N(\mu, \sigma^2)$, indique uma quantidade pivotal para:

\begin{enumerate}[a)] 

\item $\mu$, com variância conhecida;

\item $\mu$, com variância desconhecida;

\item $\sigma^2$.

\end{enumerate}
\end{exer}


\begin{exer} \rm
 Sabe-se que em indivíduos hipertensos a pressão distólica pode ser considerada com uma variável que apresenta distribuição normal com os parâmetros $\mu$ e $\sigma^2$ (ambos   desconhecidos). Uma amostra aleatória de 12 hipertensos é selecionada apresentando média de 135mmHg e s=2,4495mmHg. Encontre um intervalo com 90\% de confiança para a média populacional e para a variância populacional. Interprete os resultados.
\end{exer}


\begin{exer} \rm
 Um experimento foi conduzido para verificar se uma moeda é honesta. O experimento consistiu em arremessar esta moeda e observar o resultado. Foram observadas 179 caras em uma amostra aleatória de 400 arremessos. Encontre um IC com 99\% para $\theta$ (probabilidade de cara). Interprete o resultado encontrado,  a moeda parece ser honesta?
\end{exer}


%Exercicio Marcio - arquivo '02Lista2_Marcio.tex'
\begin{exer} \rm
Na construção de intervalos de confiança para a proporção $p$ (ou parâmetro $p$ de uma distribuição Bernoulli) podemos utilizar diferentes métodos como descrito no `Plano Aula 4' (arquivo `InferenciaB\_aula4\_topicos.pdf'). Use o R para gerar amostras ($n=10$ e depois $n=100$) Bernoulli para um determinado parâmetro p (digamos $p=0.2$) e construa os Intervalos de Confiança com coeficiente de confiança $(1-\alpha)100\%$ utilizando os três métodos. Sabe-se que ao repetir o experimento é esperado que os intervalos contenham o parâmetro em $(1-\alpha)100\%$ das vezes. Então fixe $\alpha =0.05$, repita o experimento 1000 vezes para os três métodos e verifique se em 95\% das vezes o intervalo contém o parâmetro $p=0.2$. Qual sua conclusão? Os intervalos possuem mesmo a confiança desejada?

\noindent (Obs.: Também é possível utilizar o método \textit{bootstrap} paramétrico, comentado no 'Plano Aula 5'.)
\end{exer}


\begin{exer} \rm
 Por que $H_1$ é chamada de hipótese de pesquisa?
\end{exer}


\begin{exer} \rm
 Quais os tipos de erros de um teste de hipóteses? Utilize um exemplo e indique quais os erros possíveis que um pesquisador pode cometer ao fazer um teste de hipóteses.
\end{exer}


\begin{exer} \rm
 Qual o comportamento da função poder ideal?
\end{exer}


\begin{exer} \rm
 Explique com suas palavras qual a idéia do TRV.
\end{exer}


\begin{exer} \rm
 Explique com suas palavras o que é um teste de hipóteses.
\end{exer}


\begin{exer} \rm
 Explique com suas palavras o que é um intervalo de confiança.
\end{exer}


% Exercício Marcio - exercicio_para_entregar1.tex
\begin{exer} \rm
Seja $X\in\{1,2,3,4\}$ uma variável aleatória com função massa de probabilidade $P_\theta(X=k)$, para $\theta\in\Theta=\{0,1\}$ e $k\in\{1,2,3,4\}$ dada pela seguinte tabela

\begin{center}
\begin{tabular}{c|cccc}
&$P_\theta(X=1)$&$P_\theta(X=2)$&$P_\theta(X=3)$&$P_\theta(X=4)$\\
\hline
$\theta=0$ &0.02&0.02&0.03&0.93\\
$\theta=1$ &0.10&0.20&0.30&0.40\\
\end{tabular}
\end{center}
Considere as hipóteses $H_0:\theta=0$ vs. $H_1:\theta=1$ e $X=x$ uma única observação:
\begin{enumerate}[a)] 
  \item Considere o teste A que rejeita $H_0$ se $X\leq 2$ e o teste B que rejeita $H_0$ se $X$ é par. Calcule as probabilidades $\alpha$ e $\beta$ para ambos os testes.
  \item Use o Lema de Neyman-Pearson para encontrar o teste MP para um nível de significancia de 5\%.
\end{enumerate}
\end{exer}


% Continuacao lista 2 Marcia
\begin{exer} \rm
 Faça seguintes exercícios do livro Statistical Inference: 

\begin{enumerate}[a)] 

\item 8.1 

\item 8.2

% \item 8.4
% 
% \item 8.5 (a) e (b)
% 
% \item 8.6 (a)
% 
% \item 8.7 (b)
% 
% \item 8.12
% 
% \item 8.13 (a) e (b)

\item 8.14

\item 8.15

\item 8.16
% 
% \item 8.17 (a) e (b)
% 
% \item 8.18
% 
\item 8.19

\item 8.20

\item 8.21

\item 8.22 (a)

\item 8.23 (b)

\item 8.24

\end{enumerate}
\end{exer}

%% Exercicios da lista 1 Marcia... incluir???
% \begin{exer} \rm
% Suponha que a proporção $p$ de itens defeituosos, em uma grande população de itens, 
% seja desconhecida. Deseja-se testar as seguintes hipóteses $H_0 : p = 0,2$ versus 
% $H_1 : p \neq 0,2$. Considere que uma amostra aleatória de 20 itens seja retirada 
% desta população e denote $Y$ = número de itens defeituosos na amostra. O seguinte 
% procedimento de teste será usado: Rejeitar $H_0$ se $Y \geq 7$ ou $Y \leq 1$.
% \begin{enumerate}[a)]
%   \item Determine a funcão poder deste teste.
%   \item Calcule o valor da função poder para os seguintes pontos 
%   $p = \{0, 0.1, 0.2, 0.3, 0.4, 0.5, 0.6, 0.7, 0.8, 0.9, 1\}$. Faça o gráfico.
%   \item Determine o tamanho do teste, ou seja, o valor de $\alpha = \sup_{\theta 
%   \in\Theta_0} \beta(\theta)$.
% \end{enumerate}
% \end{exer}
% 
% 
% \begin{exer} \rm
% Seja $X_1, \ldots, X_{10} $ uma amostra aleatória de tamanho $n = 10$ tal que 
% $X_i \sim Bernoulli(\theta)$ onde $P(X_i = 1) = \theta = 1 - P(X_i = 0)$. 
% Considere as hipóteses $H_0 : \theta \leq 1/2$ contra $H_1 : \theta > 1/2$. 
% Assuma a seguinte regra de teste: Rejeitar $H_0$ se $\sum X_i \geq 6$.
% \begin{enumerate}[a)]
%   \item Determine a função poder do teste.
%   \item Calcule a função poder para os seguintes pontos $p = \{0, 0.1, 0.2, 0.3, 0.4, 0.5, 0.6, 0.7, 0.8, 0.9, 1\}$. Faça o gráfico.
%   \item Determine o tamanho do teste, ou seja, o valor de $\alpha = \sup_{\theta \in\Theta_0} \beta(\theta)$.
% \end{enumerate}
% \end{exer}
% 
% 
% \begin{exer} \rm
% Considere a variável aleatória $X$ com a seguinte densidade $f(x) = \theta x^{\theta-1}I_{(0,1)}(x)$. Para testar as hipóteses $H_0 : \theta \leq 1$ versus $H_1: \theta > 1$, uma única observação $(X_1)$ foi amostrada e o seguinte critério de rejeição foi adotado: rejeitar $H_0$ se $X_1 > 1/2$.
% \begin{enumerate}[a)]
%   \item Encontre a função poder deste teste.
%   \item Determine o tamanho do teste.
% \end{enumerate}
% \end{exer}

\end{document}