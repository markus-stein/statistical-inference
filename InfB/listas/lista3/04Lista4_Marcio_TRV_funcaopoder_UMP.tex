% ----------------------------------------------------------------
% AMS-LaTeX Paper ************************************************
% **** ---------------------------------------------------------
\documentclass[10pt,brazil,addpoints]{exam}
\usepackage{geometry}
\geometry{verbose,tmargin=1.5cm,bmargin=1.2cm,lmargin=1cm,rmargin=1.5cm}
\usepackage[T1]{fontenc}
\usepackage[utf8]{inputenc}
\usepackage{babel}
\usepackage{amsmath}
\usepackage{amsfonts}
\usepackage{mathtools}
\usepackage[usenames,dvipsnames]{pstricks}
\usepackage{amssymb}
\usepackage{amsthm}

% Para redefinir o texto ao lado das pontuações:
\pointpoints{ponto}{pontos}


\usepackage{graphicx}

\usepackage{booktabs}

\usepackage{caption}

\usepackage{enumerate}

\setlength{\parindent}{0pt}

\usepackage{mleftright} % corrects spacing after \left( and before \right), etc
\mleftright

\newcommand{\E}{\mathbb{E}}
\newcommand{\Prob}{\mathbb{P}}
\newcommand{\T}{\mathbb{T}}
\newcommand{\prob}{\mathbb{P}}

\begin{document}
% Título do Artigo
\title{Lista 4 }

% Definição do(s) autor(es). Observe que é possível colocar
% uma nota de agradecimento, ou algo semelhante.

\author{
  Prof. Marcio Valk \\
  Disciplina: Inferência B\\
  %\and
  %Author 4 and Author 5  \\
  %Institution B
 % \and
  %Author 5  \\
  %Institution C
  \date{}
}

% Data. Se você quiser, pode entrar com o texto desejado no campo \date
% Caso queira a data do dia da compilação, exclua o comando \date
% Caso não queira que nada seja impresso no lugar da data, use \date{}

\maketitle


\begin{enumerate}[1.]

\item %(Pg 3 e 4 apostila) -tem resposta
Uma caixa contém 2 moedas. Uma apresenta cara com probabilidade 0,5 (equilibrada) e a outra apresenta cara com probabilidade 0,6 (viesada). Uma delas é escolhida aleatoriamente e lançada 3 vezes. Deseja-se saber se a moeda selecionada é a equilibrada ou a viesada.

\begin{enumerate}[a)]
\item Defina um teste para decidir entre $H_0:\theta=0.5$ e $H_1:\theta=0.6$.
\item Calcule as probabilidades de erro tipo I e II.
\end{enumerate}






\medskip
\item %(8.1 Casella and Berger) -tem resposta
 Em 1000 lançamentos de uma moeda, foram observadas 560 caras e 440 coroas. É razoável assumir
que a moeda é equilibrada?




\medskip
\item%(8.2 Casella and Berger) -tem resposta
 Em uma determinada cidade o número de acidentes com automóveis em dado ano segue a distribuição de Poisson.
 Nos últimos anos a média do número de acidentes por ano foi 15, e este ano foi 10. É correto afirmar que o
número de acidentes está diminuindo?






\medskip
\item%(8.13 Casella and Berger) -tem resposta
 Seja $X_1,\ldots,X_n$  i.i.d. uniforme($\theta,\theta+1$). Para testar $H_0:\theta =0$ versus (vs.) $H_1:\theta > 0$, temos dois testes concorrentes:

 $$\phi_1(X_1): \mbox{ Rejeita } H_0 \mbox{ se } X_1> 0.95,$$
 $$\phi_2(X_1): \mbox{ Rejeita } H_0 \mbox{ se } X_1+X_2> C,$$

\begin{enumerate}[a)]
\item Encontre o valor de $C$ para o qual $\phi_2$ tenha o mesmo tamanho  que $\phi_1$.

\item    Calcule a função poder de cada teste. Desenhe a função poder de cada teste.

\item $\phi_2$ é mais poderoso que $\phi_1$?

\item Mostre como encontrar um teste que tenha o mesmo tamanho de $\phi_2$, mas que seja mais poderoso que $\phi_2$.
\end{enumerate}


\medskip
\item %(ex 3 mae3-l6 listas da net) -tem resposta
 Seja $X_1,\ldots,X_n$  a.a. de uma v.a. $X$ com função de densidade dada por

\[f(x)=\theta x^{\theta-1},\,\,\,\,0<x<1,\,\,\,\theta> 0.\]

\begin{enumerate}[a)]
\item  Mostre que o teste mais poderoso para testar  $H_0:\theta =1$ vs. $H_1:\theta = 2$ rejeita $H_0$, se e somente se, $\sum_{i=1}^n -\log x_i\leq a$, em que $a$ é uma constante.

\item Sendo $n=2$ e $\alpha=(1-\log 2)/2$, qual é a região crítica?
\end{enumerate}



\medskip
\item%(ex 3 mae3-l6 listas da net) -tem resposta
 Seja $X_1,\ldots,X_n$ a.a. de uma v.a. $X$ com função de densidade $N(0,\sigma^2)$.

\begin{enumerate}[a)]
\item Encontre o teste uniformemente mais poderoso (UMP) para testar $H_0:\sigma^2 =\sigma_0^2 $ vs. $H_1:\sigma^2 > \sigma_0^2 $

\item Seja $\alpha = 0.05$, $n = 9$ e $\sigma_0^2=9$.
 Faça o gráfico da função poder.
\end{enumerate}




\medskip
\item %(8.12 Casella and Berger) -tem resposta
Para amostras de tamanho $n=1,4,16,64,100$ de uma população normal com média $\mu$ e variância conhecida $\sigma^2$, faça o gráfico da função poder dos seguintes testes da razão de verossimilhança (TRV's). Tome $\alpha=0.05$.
\begin{enumerate}[a)]
\item $H_0:\mu\leq 0$  vs. $H_1:\mu>0$.
\item $H_0:\mu= 0$  vs. $H_1:\neq 0$.
\end{enumerate}








\medskip
\item %(8.5 Casella and Berger) -tem resposta
 Uma a.a. $X_1,\ldots,X_n$ é retirada de uma população Pareto com densidade

$$f(x|\theta,\nu)=\frac{\theta\nu^\theta}{x^{\theta+1}}I_{\nu,\infty}(x),\,\,\,\theta>0,\,\,\,\nu>0.$$

\begin{enumerate}[a)]
\item Encontre os EMV's de $\theta$ e $\nu$.

\item Mostre que o TRV
$$H_0: \theta=1,\,\,\nu\,\,\mbox{desconhecido},\,\,\,\,\,\,\,vs\,\,\,\,\,\,\, H_1: \theta\neq 1,\,\,\nu\,\,\mbox{desconhecido},$$
tem região critica da forma $\{x:T(x)\leq c_1\,\,ou\,\,T(x)\geq c_2\}$, em que $0<c_1<c_2$ e $$T=\log \left[ \frac{\prod_{i=1}^{n}X_i}{(\min_i X_i)^n}\right].$$
\end{enumerate}





\medskip
\item %(8.6 Casella and Berger) -tem resposta
Suponhamos que temos duas amostras de variáveis aleatórias independentes: $X_1\ldots,X_n$ são exp($\theta$) e $Y_1\ldots,Y_n$ são exp($\mu$). Encontre o TRV de $H_0:\theta=\mu$ vs. $H_1:\theta\neq \mu$.

%\begin{enumerate}[a)]
%\item Encontre o TRV de $H_0:\theta=\mu$ vs. $H_1:\theta\neq \mu$.
%\item Mostre que o teste da parte a) pode ser baseado na estatística
%$$T=\frac{\sum X_i}{\sum X_i+\sum Y_i}.$$
%\item Encontre a distribuição de $T$ quando $H_0$ é verdadeira.
%
%\end{enumerate}









\medskip
\item %(8.8 Casella and Berger) -tem resposta
Um caso especial da família de distribuições \emph{normal} é quando a média e a variância são relacionadas, como por exemplo a família $N(\theta,a\theta)$. Se estamos interessados em testar esse relacionamento, independente do valor de $\theta$, nos deparamos com um problema chamado problema do parâmetro ``nuisance''.

\begin{enumerate}[a)]
\item Encontre o TRV de $H_0:a=1$ vs. $H_1:a\neq 1$ baseado em uma amostra $X_1,\ldots,X_n$ de uma família $N(\theta,a\theta)$, em que $\theta$ é desconhecido.

\item  Um problema similar ocorre quando a família é $N(\theta,a\theta^2)$. Assim, se  $X_1,\ldots,X_n$ são i.i.d. $N(\theta,a\theta^2)$, quando $\theta$ é desconhecido, encontre o TRV de $H_0:a=1$ vs. $H_1:a\neq 1$.
\end{enumerate}






\end{enumerate}

\end{document}
% ----------------------------------------------------------------
