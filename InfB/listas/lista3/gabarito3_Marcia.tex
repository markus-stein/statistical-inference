% ----------------------------------------------------------------
% AMS-LaTeX Paper ************************************************
% **** ---------------------------------------------------------
\documentclass[10pt,brazil,addpoints]{exam}
\usepackage{geometry}
\geometry{verbose,tmargin=1.5cm,bmargin=1.2cm,lmargin=1cm,rmargin=1.5cm}
\usepackage[T1]{fontenc}
\usepackage[utf8]{inputenc}
\usepackage{babel}
\usepackage{amsmath}
\usepackage{amsfonts}
\usepackage{mathtools}
\usepackage{breqn}
\usepackage[inline]{enumitem}

\usepackage[usenames,dvipsnames]{pstricks}
%\usepackage{epsfig}
\usepackage{pst-grad} % For gradients
\usepackage{pst-plot} % For axes

\pagestyle{plain}

% Para redefinir o texto ao lado das pontuações:
\pointpoints{ponto}{pontos}
% Para redefinir as margens:
%\extrawidth{1in}
%\extraheadheight{-.5in}

\usepackage{graphicx}

\usepackage{booktabs}

\usepackage{caption}

\usepackage{enumerate}

\setlength{\parindent}{0pt}

\usepackage{mleftright} % corrects spacing after \left( and before \right), etc
\mleftright

\newcommand{\E}{\mathbb{E}}
\newcommand{\Prob}{\mathbb{P}}

\begin{document}
% Título do Artigo
\title{GABARITO Lista 3}

% Definição do(s) autor(es). Observe que é possível colocar
% uma nota de agradecimento, ou algo semelhante.

\author{
  Profª. Márcia Barbian \\
  Disciplina: Inferência B\\
  %\and
  %Author 4 and Author 5  \\
  %Institution B
 % \and
  %Author 5  \\
  %Institution C
  \date{}
}

% Data. Se você quiser, pode entrar com o texto desejado no campo \date
% Caso queira a data do dia da compilação, exclua o comando \date
% Caso não queira que nada seja impresso no lugar da data, use \date{}

\maketitle


\begin{enumerate}[1.]

\item 
Lembrando que para calcular o método de Newton-Raphson utilizamos a seguinte equação:

\begin{flushleft}
\theta^{(n+1)}=\theta^{(n)}-\frac{\ell'(\theta^{(n)})}{\ell''(\theta^{(n)})}
\end{flushleft}

Os valores das derivadas de primeira e de segunda ordem do log da função de verossimilhança $(\ell(\theta))$ são respectivamente:

\begin{flushleft}
\ell'(\alpha)=\frac{n}{\alpha}-\frac{n\alpha\sum X_i^{\alpha}\log X_i}{\sum X_i^{\alpha}}+ \sum \log X_i
\end{flushleft}

\begin{flushleft}
\ell''(\alpha)=\frac{n}{\alpha^2}+n\frac{\sum X_i^{\alpha}\sum X_i^{\alpha} (\log X_i)^2-(\sum X_i^{\alpha}\log X_i)^2}{(\sum X_i^{\alpha})^2}
\end{flushleft}

Agora é substituir $\theta$ por $\alpha$ e considerar como primeiro passo $\alpha=0.5$.

\item Para calcular o método de Newton-Raphson precisamos da derivada de primeira ordem $g'(\theta)=2\theta$. O algoritmo é calculado da seguinte forma:
\begin{flushleft}
\theta^{(n+1)}=\theta^{(n)}-\frac{g(\theta^{(n)})}{g'(\theta^{(n)})}
\end{flushleft}

\begin{flushleft}
\theta^{(n+1)}=\theta^{(n)}-\frac{\theta^{(n)^2}-4}{2\theta^{(n)}}
\end{flushleft}



\item 

\item Se $\theta_0< \overline{X}$, então 

\begin{flushleft}
\lambda(X)=\left(\frac{\theta_0}{\overline{X}}\right)^{\sum X_i}\left(\frac{1-\theta_0}{1-\overline{X}}\right)^{n-\sum X_i}
\end{flushleft}

\item 
\begin{enumerate}[a)]
\item Rejeitar $H_0$ se $
((\overline{X})^n\exp\{n(1-\overline{X})\})< c $ para algum $c \in (0,1)$

\item Não rejeitar $H_0$.
\end{enumerate}

\item  
\begin{enumerate}[a)]
\item Rejeitar $H_0$ se $\exp\{-\frac{1}{2V}W^2\}< c$, onde $V=\frac{9}{n}+\frac{25}{m}$, $W=\overline{X}-\overline{Y}$ e $c\in (0,1)$.

\item Não rejeitar $H_0$
\end{enumerate}


\item Se $\lambda_0 < \hat{\lambda}$ rejeitamos $H_0$ se $\left(\left(\frac{\lambda_0}{\hat{\lambda}}\right)^{3n}\exp{(\hat{\lambda}-\lambda_0)\sum X_i}\right)< c$, onde $c \in (0,1)$ e $\hat{\lambda}$ é o EMV de $\lambda$.

\item Se $\theta_0 \geq x_{(n)}$ rejeitamos $H_0$ se $\left(\frac{x_{(n)}}{\theta_0}\right) < c$ onde $c \in (0,1)$ e $x_{(n)}=\max\{X_1, \ldots, X_n\}$


\end{enumerate}




\end{document}
% ----------------------------------------------------------------
