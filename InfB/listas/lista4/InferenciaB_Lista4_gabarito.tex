\documentclass[letter,11pt]{article}

\usepackage{amsfonts}
\usepackage{amsmath}
\usepackage{amssymb}
\usepackage[brazilian]{babel}
\usepackage{enumerate}
\usepackage[T1]{fontenc}
%\usepackage[ansinew,latin1]{inputenc}
\usepackage[utf8x]{inputenc}
\usepackage{multicol}
\setlength\columnseprule{0.5pt}

\newtheorem{exer}{Exercício}
\newtheorem{teo}{Teorema}

\newcommand{\var}{Var}
\newcommand{\E}{\mathbb{E}}

\newcommand{\mat}[1]{\mbox{\boldmath{$#1$}}}

\usepackage[letterpaper,top=3cm, bottom=2cm, left=2.5cm, right=2.5cm]{geometry}

\begin{document}

%\thispagestyle{empty}
\begin{center}{ \Large MAT02026 - Inferência B }\end{center}

\begin{center}
{\large  \sc Gabarito Lista 4 - TRV assintótico, IC como testes, valor $p$ e RVM}
\end{center}
\vspace{5mm}

%%%%%%%%%%%%%%%%%%%%%%%%%%%%%%%%%%%%%%%%%%%%%%%%%%%%%%%%%%%%%%%%%%%%%%%%%%%%%%%
% Exercicio estendido prova 1
\begin{exer} \rm
%Seja $X_1$ uma única observação obtida da distribuição $Beta(\theta, 1)$
\begin{enumerate}[a)]
  \item %Mostre que $X_1^{\theta}$ é uma quantidade pivotal.
  Resolvido no gabarito da prova 1.
  
  \item %Construa um intervalo com 95\% de confiança utilizando a quantidade 
  % pivotal $X_1^{\theta}$.
  Resolvido no gabarito da prova 1.
  
  \item %Assuma *a piori* $\theta \sim Gama(\alpha=1, \beta)$, encontre um 
  %intervalo $1 - \alpha$ de credibilidade para $\theta$. Compare os intervalos.
  Resolvido no gabarito da prova 1. 
  Comentar sobre confiança $\times$ credibilidade.
  
  \item %Comente sobre as suposições para construirmos intervalos segundo as 
  %duas abordagens.
  Falar das suposições de modelo e amostragem.

  \item %Teste de hipóteses frequentistas e bayesianos também podem ser 
  % construídos com base nos intervalos de confiança e de credibilidade, 
  % respectivamente. Gere uma amostra de tamanho $n=1$ da distribuição 
  % $Beta(1,5; 1)$ e teste a hipótese $H_0 : \theta = 1$ contra $H_1: \theta \neq 1$.
  Em breve...

  \item %Calcule o valor $p$ para os testes acima. Justifique os cálculos e interprete os valores $p$.
  Em breve...
  
\end{enumerate}
\end{exer}


\begin{exer} \rm
%Quiz sobre valor $p$.  
\begin{enumerate}[a)]
  \item %Qual o significado do valor $p$ na prática? Como a ciência tem utilizado o valor $p$ para criar suas teorias? Cite exemplos. 
  \item %Porque o uso do valor $p$ tem sido muito criticado recentemente? 
  \item %Qual sua conclusão sobre o problema. Indique alternativas ao valor $p$.
\end{enumerate}
\end{exer}

%%%%%%%%%%%%%%%%%%%%%%%%%%%%%%%%%%%%%%%%%%%%%%%%%%%%%%%%%%%%%%%%%%%5%%%%%%%%%%%
% Lista 5 marcia

%Questão 10.35 (pag 513) do Casela
\begin{exer} \rm 
%Seja $X_1, \ldots, X_n$ uma amostra aleatória de uma população com distribuição $N(\mu, \sigma^2)$

\begin{enumerate}[a)]

\item %Se $\mu$ é desconhecido e $\sigma^2$, mostre que $Z=\sqrt{n}(\overline{X}-\mu_0)/\sigma$ é um teste de Wald para $H_0:\mu=\mu_0$.

Uma vez que $\frac{\sigma}{\sqrt n}$ é o desvio padrão estimado (ou erro padrão?) de $\overline X$ , então a estatística é uma estatística do tipo Wald.

\item %Se $\sigma^2$ é desconhecido e $\mu$ é conhecido, encontre a Estaistica de Wald para testar $H_0:\sigma=\sigma_0$.

O EMV de $\sigma^2$ para um valor fixado de $\mu$ é dado por $\hat \sigma^2_\mu = \frac{\sum_i (x_i - \mu)^2}{n}$. 

(* Obs.: Se $W_n$ é EMV, $1 / \sqrt{I_n(W_n)}$ é um erro padrão para $W_n$. Alternativamente, usamos $1 / \sqrt{\hat{I}_n(W_n)}$) em que $\hat{I}_n(W_n)$ é a informação de Fisher observada, $\left. \hat{I}_n(W_n) = - \frac{\partial^2}{\partial \theta^2} \log L(\theta) \right|_{\theta = W_n}$).

A informação observada de Fisher é dada por 

$$\left. - \frac{\partial^2}{\partial (\sigma^2)^2} \left(- \frac{n}{2} \log \sigma^2 - \frac{\hat \sigma^2_\mu}{2 \sigma^2} \right) \right|_{\sigma^2 = \hat \sigma^2_\mu} = \frac{n}{2 \hat \sigma^2_\mu}.$$

Então, usando o Método Delta, a variância de $\hat \sigma_\mu = \sqrt{\hat \sigma^2_\mu}$ é dada por $Var(\hat \sigma_\mu) = \frac{\hat \sigma^2_\mu}{8n}$, então uma estatística do tipo Wald é dada por

$$\frac{\hat \sigma_\mu - \sigma_0}{\sqrt{\frac{\sigma^2_\mu}{8n}}}.$$

\end{enumerate}
\end{exer}


%questão 17 pag 259 do migon
\begin{exer} \rm
%Seja $X_1, \ldots, X_n$ uma amostra aleatória da distribuição Exponencial $(\theta)$, suponha que queremos testar $H_0:\theta=1$.

\begin{enumerate}[a)]

\item %Mostre que o TRV rejeita $H_0$ quando $\sum X_i<c$. 
Para o TRV precisamos calcular $\lambda(\boldsymbol{x})$. Note que, como o $H_0$ é hipótese simples, o EMV restrio é dado por $\hat \theta_0 = 1$ e o EMV irrestrito é $\hat \theta = 1 / \overline{X}$ (verificar!). Assim
$$ \lambda(\boldsymbol{x}) = \frac{\sup_{\theta \in \Theta_0} L(\theta)}{\sup_{\theta \in \Theta} L(\theta)} = \frac{L(\hat \theta_0)}{L(\hat \theta)}  = \frac{\hat \theta_0^n e^{- \hat \theta_0 \sum_{i=1}^n x_i}}{\hat \theta^n e^{- \hat \theta \sum_{i=1}^n x_i}}= \frac{\frac{1}{1}^n e^{- \frac{1}{1} \sum_{i=1}^n x_i}}{\frac{1}{\overline x}^n e^{- \frac{1}{\overline x} \sum_{i=1}^n x_i}} = \frac{e^{- n \overline{x}}}{\frac{1}{\overline x}^n e^{- n}}.$$

Então, a região de rejeicao $A_1$ para testar $H_0:\theta \leq 1$ pode ser reescrita como 

$$ A_1 = \left\{ \lambda(\boldsymbol{x}) < c \right\} \Leftrightarrow A_1 = \left\{ \overline{x}^n e^{-\overline{x} n} < c^* \right\}.$$

A segunda forma de $A_1$ é claramente uma função de $\widebar X$ e usando propriedades da função $Gama$ podemos mostrar também que $\sum_{i=1}^n X_i \leq c^{**}$ 

\item %Qual o valor de $c$ para $\alpha=0.05$. 
Verifique que $Y = \sum_{i=1}^n X_i \sim Gama(n, \theta)$, então basta encontrar $c$ tal que $$ 0,05 = P(Y < c) $$.


\item %Construa o teste assintótico da razão de verossimilhança e o teste de Wald e compare com o teste exato.
TRV assintótico:

Para o teste assintótico usamos o resultado (Teorema de Wilks) 
$$\lambda(\boldsymbol{x}) \sim \chi^2_{(1)},$$ 
(porque 1 grau de liberdade?) quando $n \rightarrow \infty$. Assim $ Z =  \overline{x}^n e^{-\overline{x}} \sim \chi^2_{(1)}$ quando o tamanho da amostra cresce indefinidamente. ...


Teste de Wald:

Usando as propriedades do EMV, temos que sob $H_0$ $\frac{\widehat \theta - \theta_0}{\sqrt{Var(\widehat \theta)}} \sim Normal(0, 1)$, quando $n \rightarrow \infty$. Lembrando que, sob certas condições de regularidade, $Var(\widehat \theta) = I_n(\theta)$ (vide exercício 3). ...


\item %Gere aleatoriamente uma amostra de $n=20$ e $\theta=1.5$ de uma distribuição exponencial. Calcule os testes de Wald, verosmilhança assintótico e teste de verossimilhança exato para essa amostra. Repita o experimento 100 vezes e indique quantas vezes rejeita-se a $H_0$ a um nível de significância de 5\%. Compare os resultados.
Em breve...

\end{enumerate}
\end{exer}


%exercicio 12 wasserman pag 173
\begin{exer} \rm
%Seja $X_1, \ldots, X_n$ uma amostra aleatória de uma distribuição de Poisson ($\lambda$).
  \begin{enumerate}[a)]
    \item %Seja $\lambda_0>0$, encontre um teste de Wald de tamanho $\alpha$ para
% $$
% H_0: \lambda=\lambda_0  \mbox{ versus } H_1: \lambda \neq \lambda_0
% $$

    \item %Calcule o TRV para as hipóteses acima.

    \item %Calcule o TRV assintótico para as hipóteses acima. 
  
    \item %Calcule o Teste de Wald para as hipóteses acima.

    \item %Seja $\lambda=1$ e $\lambda_0=0.8$, $n=20$ e $\alpha=0.05$. Simule $X_1, X_2, \ldots, X_n$ de uma distribuição de Poisson $(\lambda)$ e calcule os testes acima. Repita esse procedimento 100 vezes, calcule quantas vezes a hipótese nula foi rejeitada. Qual o valor do do erro tipo I para cada um dos testes?
\end{enumerate}
\end{exer}


\begin{exer} \rm
%Verifique se as as distribuicoes dos exercícios 1,2 e 3 possuem razão de verossimilhança monótona. 
\end{exer}


\begin{exer} \rm
%Construa o teste uniformemente mais poderoso de tamanho $\alpha$ para:

  \begin{enumerate}
    \item %Exercício 1 letra a, em que $H_0: \mu \leq \mu_0$.
    
    \item %Exercício 2, em que $\theta \leq 1$.
    
    \item %Exercício em que $\lambda \leq \lambda_0$.
  \end{enumerate}
\end{exer}

\end{document}