% ----------------------------------------------------------------
% AMS-LaTeX Paper ************************************************
% **** ---------------------------------------------------------
\documentclass[10pt,brazil,addpoints]{exam}
\usepackage{geometry}
\geometry{verbose,tmargin=1.5cm,bmargin=1.2cm,lmargin=1cm,rmargin=1.5cm}
\usepackage[T1]{fontenc}
\usepackage[utf8]{inputenc}
\usepackage{babel}
\usepackage{amsmath}
\usepackage{amsfonts}
\usepackage{mathtools}
\usepackage{breqn}
%\usepackage[inline]{enumitem}

\usepackage[usenames,dvipsnames]{pstricks}
%\usepackage{epsfig}
\usepackage{pst-grad} % For gradients
\usepackage{pst-plot} % For axes

\pagestyle{plain}

% Para redefinir o texto ao lado das pontuações:
\pointpoints{ponto}{pontos}
% Para redefinir as margens:
%\extrawidth{1in}
%\extraheadheight{-.5in}

\usepackage{graphicx}

\usepackage{booktabs}

\usepackage{caption}

\usepackage{enumerate}

\setlength{\parindent}{0pt}

\usepackage{mleftright} % corrects spacing after \left( and before \right), etc
\mleftright

\newcommand{\E}{\mathbb{E}}
\newcommand{\Prob}{\mathbb{P}}

\begin{document}
% Título do Artigo
\title{Lista 7}

% Definição do(s) autor(es). Observe que é possível colocar
% uma nota de agradecimento, ou algo semelhante.

\author{
  Profª. Márcia Barbian \\
  Disciplina: Inferência B\\
  %\and
  %Author 4 and Author 5  \\
  %Institution B
 % \and
  %Author 5  \\
  %Institution C
  \date{}
}

% Data. Se você quiser, pode entrar com o texto desejado no campo \date
% Caso queira a data do dia da compilação, exclua o comando \date
% Caso não queira que nada seja impresso no lugar da data, use \date{}

\maketitle

\begin{enumerate}[1.]


\item  Considere que $X_1, \ldots, X_n$ é uma amostra aleatória de uma população com distribuição Binomial Negativa $(r, \theta)$, $r>0$ e $0<\theta<1$, cuja função de probabilidade é dada por:

$$
p(x/ r, \theta) = {x+r-1 \choose x}\theta^r(1-\theta)^x
$$

Assuma que $r$ é conhecido

\begin{enumerate}[a)]

\item Usando a distribuição conjugada natural de $\theta$, obtenha a distribuição a \textit{posteriori} de $\theta$.

\item Suponha que foi selecionada uma amostra aleatória dessa população e que obteve $\sum x_i=70$, sendo $r=5$ e $n=10$. Considere as hipóteses:

$H_0: \theta \leq 0.5$ vs $H_1: \theta > 0.5$

Qual das hipóteses apresenta uma maior probabilidade a \textit{posteriori}, admitindo que possui uma informação ``moderada" acerca de $\theta$, a qual é introduzida no modelo por meio de uma distribuição Beta (2,2)?

\item Indique o valor do fator de bayes em favor de $H_1$. Poderá concluir que existe forte evidência em favor desta hipótese?

\end{enumerate}

\item Dezesseis consumidores habituais de comida de uma cadeia de fast food foram recrutados
para participar de um estudo cujo objectivo era comparar o sabor de dois tipos de recheio usado num certo tipo de bolo confeccionado com carne de vaca. Um
dos conjuntos de dezasseis recheios que foi analisado tinha sido congelado 8 meses
atrás e mantido num congelador de elevada qualidade (com alterações de temperatura de menos do que um grau na escala de Fahrenheit) durante todo o período de
tempo. Os restantes 16 recheios foram armazenados em congeladores de qualidade média, com variações de temperatura entre 0 e 15 graus Fahrenheit. 

Os gestores da
cadeia de fast food pretendem verificar se a qualidade do refrigeramento altera o sabor
dos recheios e portanto avaliar se valerá a pena o elevado custo associado a uma maior qualidade de refrigeração.

Os produtos são descongelados e cozinhados por um único chef. O planejamento do experimento é feito de forma a que seja “duplamente-cego”, i.e., nem os empregados
que servem o recheio nem os consumidores têm qualquer conhecimento sobre
os produtos que estão a servir (consumir) em qualquer momento. No fim da experiência
observa-se que 13 dos 16 consumidores preferiram os recheios que tinham
sido congelados no congelador de maior qualidade.


\begin{enumerate}[a)]
    \item Qual é o modelo que você sugere para analisar estes dados? Que pressupostos deve impor?

    \item Seja $\theta$ a probabilidade de que os consumidores prefiram o produto mais caro. Sejam Beta(0.5, 0.5), Beta(1.0, 1.0) e Beta(2.0, 2.0) três distribuições que refletem diferentes níveis de conhecimento, a priori, relativamente a $\theta$. Obtenha as distribuições a posteriori correspondentes. Para cada uma delas,
    \begin{itemize}
        \item calcule a média, moda e mediana de $\theta$ a posteriori. Comente os resultados que obtiver.
        \item  Indique a $P(\theta > 0.6 | x)$.
        \item Calcule o intervalo HPD de 95\% de credibilidade para $\theta$.
        \item Calcule o fator de Bayes para $H_0 : \theta \geq 0.6$ vs. $H_1 : \theta < 0.6$. Faça os comentários que considerar adequados.
    \end{itemize}
    
\end{enumerate}

\item Sejam $X_1, X_2,\ldots, X_n$ variáveis aleatórias i.i.d. com distribuição de Poisson($\theta$), $\theta> 0$. Considere a distribuição Gama(1, 1) como distribuição a priori para $\theta$. Obtenha o intervalo de credibilidade de 90\% de igual probabilidade e o intervalo de
credibilidade HPD de 90\% para $\theta$. Considerando uma amostra aleatória
de dimensão $n = 10$ para a $\sum x_i = 6$.

\item Numa certa população, seja $\theta$ a proporção de indivíduos que têm uma determinada doença. Suponha que se pretende testar a hipótese $H_0 : \theta = 0.2$ contra a hipótese alternativa $H_1 : \theta = 0.3$. Informações anteriores mostram que $P(H_0) = 0.25$. Suponha que se observam $n$ indivíduos verificando-se que $x$ apresentam a doença.
Calcule o fator de Bayes a favor de $H_0$. Para que valores de $x$ é que se tem $P(H_0 | x) > P(H_1 | x)$?


\item Ler o material do blog https://www.countbayesie.com/blog/2016/3/16/bayesian-reasoning-in-the-twilight-zone

%13 vanessa
\item  Deseja-se estimar $\theta$: a proporção de residentes de determinada cidade que concordam com a construção de um presídio na cidade. Para isto, observou-se uma amostra de 100 pessoas, das quais 26 concordavam com a construção do presídio.

\begin{enumerate}[a)]
\item Antes de observar a amostra, um especialista afirmou que essa proporção se comportava conforme uma distribuição Beta com esperança a priori de 0,20 e variância a \textit{priori} de 0,0064. Encontre os parâmetros desta priori e construa a \textit{posteriori}.

\item Calcule a estimativa de MVG e de Bayes.
\item  Através do R, encontre o ICs 95\% Central e HPD.
\item Seja $H_0 : \theta \leq  0, 5$. Qual a probabilidade desta hipótese ser verdadeira, a \textit{posteriori}?
\end{enumerate}

%14 vanessa
\item Deseja-se estimar o diâmetro médio de determinada peça produzida em certa fábrica. A experiência indica que esse diâmetro é normalmente distribuído com variância 4$cm^2$.
Um engenheiro escolheu como priori para $\mu$ uma $N(\mu_0 = 30; \sigma_0^2=16)$, pois acredita ser praticamente impossível que esse diâmetro seja menor que 18 cm ou maior que 42 cm. Uma amostra de 12 peças foi observada e se obteve $\overline{x}=32$cm.

\begin{enumerate}[a)]
    \item Calcule o IC 95\% HPD para $\mu$. Calcule também o 1º
e 3º quartis.
\item Usando priori $\pi(\mu)$ propocional a 1, repita o item a.
\item Qual seria o IC 95\% encontrado pela inferência clássica? Compare com os itens anteriores.

\end{enumerate}
\end{enumerate}


\end{document}
% ----------------------------------------------------------------
