% ----------------------------------------------------------------
% AMS-LaTeX Paper ************************************************
% **** ---------------------------------------------------------
\documentclass[10pt,brazil,addpoints]{exam}
\usepackage{geometry}
\geometry{verbose,tmargin=1.5cm,bmargin=1.2cm,lmargin=1cm,rmargin=1.5cm}
\usepackage[T1]{fontenc}
\usepackage[utf8]{inputenc}
\usepackage{babel}
\usepackage{amsmath}
\usepackage{amsfonts}
\usepackage{mathtools}

% Para redefinir o texto ao lado das pontuações:
\pointpoints{ponto}{pontos}
% Para redefinir as margens:
%\extrawidth{1in}
%\extraheadheight{-.5in}

\usepackage{graphicx}

\usepackage{booktabs}

\usepackage{caption}

\usepackage{enumerate}

\setlength{\parindent}{0pt}


\newcommand{\E}{\mathbb{E}}
\newcommand{\Prob}{\mathbb{P}}

\begin{document}
% Título do Artigo
\title{GABARITO Lista 1}

% Definição do(s) autor(es). Observe que é possível colocar
% uma nota de agradecimento, ou algo semelhante.

\author{
  Prof. Marcio Valk \\
  Disciplina: Inferência B\\
  %\and
  %Author 4 and Author 5  \\
  %Institution B
 % \and
  %Author 5  \\
  %Institution C
  \date{}
}

% Data. Se você quiser, pode entrar com o texto desejado no campo \date
% Caso queira a data do dia da compilação, exclua o comando \date
% Caso não queira que nada seja impresso no lugar da data, use \date{}

\maketitle


\begin{enumerate}[1.]
\medskip
\item Defina $A=\{L(x)\leq \theta\}$ e $B=\{U(x)\geq\theta\}$.
Verifique que $\Prob(A\cup B)=1$ e $\Prob(A\cap B)=\Prob(L(x)\leq\theta\leq U(x)).$ 

\medskip


%%Exe 2
\item 

\begin{enumerate}[a)]
\item % 9.3 do Casela
% o EMV está no exercicio 7.10 do Casela
Use que o EMV para $\beta$ é $X_{(n)}=max(X_i)$ . Como sabemos a distribuição de cada $X_i$, sabemos também a do max$(X_i)$. Assim, $X_{(n)}/\beta$ é uma quantidade pivotal e 
\[0.05=\Prob\left(\frac{X_{(n)}}{\beta}\leq q\right)
=\Prob\left(X_{(n)}\leq q\beta\right)=\left(\frac{q\beta}{\beta}\right)^{\alpha_0n}=q^{\alpha_0n}\] 

Assim, $q=0.05^{-\alpha_0n}$ e  $0.95=\Prob\left(\frac{X_{(n)}}{\beta}> q\right)=\Prob\left(\frac{X_{(n)}}{q}> \beta\right)$. Portanto 

$[\frac{X_{(n)}}{0.05^{-\alpha_0n}}]$ é a cota superior desejada.
\item Use $\hat{\alpha}=\frac{1}{n}\left[\sum_i\left( \log(X_{(n)}-\log(X_i)\right)\right]^{-1}$. $\hat{\alpha}= 12.59$ e  $X_{(n)} = 25$. Assim o IC fica $(0, 25.43)$.
\end{enumerate}

%%EXE 3
\medskip
\item 
\begin{enumerate}[a)]
\item  $X-\theta\sim U[-1/2,1/2]$. Assim $\Prob(q_1<X-\theta<q_2)=q_2-q_1$. Logo escolhendo $q_2=1/2-\frac{\alpha}{2}$ e $q_1=-1/2+\frac{\alpha}{2}$ segue que o IC dado por  $[X-1/2+\frac{\alpha}{2};X+1/2-\frac{\alpha}{2}]$ tem $(1-\alpha)$100\%  de confiança para $\theta$. 
\item Observe que $X/\theta$ tem densidade $f(t)=2t$, $0\leq t\leq 1$. Assim  $\Prob(q_1<X/\theta<q_2)=q_2^2-q_1^2$. Escolha $q_1=\sqrt{\alpha/2}$ e $q_2=\sqrt{1-\alpha/2}$.
\end{enumerate}


%%EXE4
\medskip
\item
 Use $\frac{\overline{X}-\theta}{S/\sqrt{n}}$ como pivô. (É possível usar outro pivô?)


%%EXE5
\medskip
\item
\begin{enumerate}[a)]
\item $Y=-((log(X))^{-1}$, $f_Y(y)=\frac{\theta}{y^2}e^{-\theta/y}$, $0<y<\infty$. Assim $\Prob(Y/2\leq \theta\leq Y)=\int_{\theta}^{2\theta}f_Y(y)dy=0.239$
\item $X^\theta \sim U(0,1)$. Assim, $\Prob(q_1\leq X^\theta\leq q_2)=q_2-q_1$. 
\item $\left[\frac{\log(0.975)}{\log X} ; \frac{\log(0,025)}{\log X}\right]$.
\end{enumerate}


%%EXE6
\medskip
\item $\left[\overline{X}-1,96\sqrt{1/n};  \overline{X}+1,96\sqrt{1/n}\right]$


%% EXE7
\medskip
\item 
\begin{enumerate}[a)]
\item $\lambda\sum_{i=1}^{10} X_i \sim Gama(10,1)$.

\item $\left[\frac{5,425}{\sum_{i=1}^{10} X_i} ; \frac{15,705}{\sum_{i=1}^{10} X_i}\right]$

\item $2\lambda\sum_{i=1}^{10} X_i \sim Gama(10,1/2)$.
\end{enumerate}
%
%
%\item 
%\begin{enumerate}[a)]
%\item $\beta(p)$ inclui as seguintes expressões:
%$$
%\sum_{y=7}^{20}  {20 \choose y}p^y(1-p)^{20-y} \qquad e \qquad \sum_{y=0}^1  {20 \choose y}p^y(1-p)^{20-y}
%$$
%
%
%\item $\beta(p) = \{1, 0.39, 0.16, 0.4, 0.75, 0.94, 0.99, 1, 1, 1, 1\}$. 
% 
%\end{enumerate}
%
%\item 
%\begin{enumerate}[a)]
% \item $\beta(p)$ inclui a expressão:
%$$
%\sum_{x=6}^{10}  {10 \choose x}\theta^x(1-\theta)^{10-x} 
%$$
% 
% \item $\beta(p) = \{0, 0, 0.01, 0.05, 0.17, 0.38, 0.63, 0.85, 0.97, 0.99, 1\}$. 
% 
% \item 0,38
%\end{enumerate}
%
%\item 
%
%\begin{enumerate}[a)]
% \item $\beta(\theta)=1-\frac{1}{2^\theta}$
% \item 0.5
%\end{enumerate}



\end{enumerate}




\end{document}
% ----------------------------------------------------------------
