% ----------------------------------------------------------------
% AMS-LaTeX Paper ************************************************
% **** ---------------------------------------------------------
\documentclass[10pt,brazil,addpoints]{exam}
\usepackage{geometry}
\geometry{verbose,tmargin=1.5cm,bmargin=1.2cm,lmargin=1cm,rmargin=1.5cm}
\usepackage[T1]{fontenc}
\usepackage[utf8]{inputenc}
\usepackage{babel}
\usepackage{amsmath}
\usepackage{amsfonts}
\usepackage{mathtools}
\usepackage[usenames,dvipsnames]{pstricks}
\usepackage{amssymb}
\usepackage{amsthm}

% Para redefinir o texto ao lado das pontuações:
\pointpoints{ponto}{pontos}


\usepackage{graphicx}

\usepackage{booktabs}

\usepackage{caption}

\usepackage{enumerate}

\setlength{\parindent}{0pt}

\usepackage{mleftright} % corrects spacing after \left( and before \right), etc
\mleftright

\newcommand{\E}{\mathbb{E}}
\newcommand{\Prob}{\mathbb{P}}
\newcommand{\T}{\mathbb{T}}
\newcommand{\prob}{\mathbb{P}}

\begin{document}
% Título do Artigo
\title{Lista 1}

% Definição do(s) autor(es). Observe que é possível colocar
% uma nota de agradecimento, ou algo semelhante.

\author{
  Prof. Marcio Valk \\
  Disciplina: Inferência B\\
  %\and
  %Author 4 and Author 5  \\
  %Institution B
 % \and
  %Author 5  \\
  %Institution C
  \date{}
}

% Data. Se você quiser, pode entrar com o texto desejado no campo \date
% Caso queira a data do dia da compilação, exclua o comando \date
% Caso não queira que nada seja impresso no lugar da data, use \date{}

\maketitle


\begin{enumerate}[1.]

\item Se $L(X)$ e $U(X)$ satisfazem $\prob_\theta(L(X)\leqslant\theta)=1-\alpha_1$, $\prob_\theta(U(X)\geqslant\theta)=1-\alpha_2$ e $L(x) \leqslant U(x)$ para todo $x$, mostre que

\[\prob_\theta(L(X)\leqslant\theta\leqslant U(X))=1-\alpha_1-\alpha_2.\]




\item 
Seja $X_1,\cdots,X_n$ uma a.a.com função de distribuição comum

\[\prob(X_i\leqslant x)=\left\{
                          \begin{array}{ll}
                            0, & \hbox{se }x\leqslant0, \\
                            \left(\frac{x}{\beta}\right)^\alpha, & \hbox{se }0<x<\beta, \\
                            1, & \hbox{se }x\geqslant\beta.
                          \end{array}
                        \right.
\]

\begin{enumerate}[\bf (a)]
\item Se $\alpha$ é uma constante conhecida, $\alpha_0$, encontre um limite de confiança superior para $\beta$ com coeficiente de confiança de $95\%$.

\item Utilize os dados
%\begin{center}
\begin{tabular}{cccccccccccccc}
 22,0 & 23,9 & 20,9 & 23,8 & 25,0 & 24,0 & 21,7 & 23,8 & 22,8 & 23,1 & 23,1 & 23,5 & 23,0 & 23,0 \\
\end{tabular}
%\end{center}

\noindent para construir uma estimativa de intervalo para $\beta$. Assuma que $\alpha$ é conhecido e igual a estimativa gerada pelo seu estimador de máxima verossimilhança utilizando os dados acima.
\end{enumerate}

% Casela 9.17
 \item  Seja $X_1,\cdots,X_{n}$ uma a.a. onde cada uma das variáveis aleatórias possui função densidade de probabilidade dada abaixo. Encontre um intervalo de confiança para $\theta$ com coeficiente de confiança $1-\alpha$.

\begin{enumerate}[\bf (a)]
\item $f(x)=1$, para $\theta-\frac{1}{2}<x<\theta+\frac{1}{2}$.

\item $f(x)=\frac{2x}{\theta^2}$, para $0<x<\theta$, onde $\theta>0$.
\end{enumerate}

%%%%%%%%%%%%%%%%%%%%%%%%%%%%%%%%%%%%%%%%%%%%%%%%%%%%%%%%%%%%%%%%%%%%%%%%%%%%%%%%%%%%
% Casela 9.12
\item Encontre uma quantidade pivotal com base em uma amostra aleatória de tamanho $n$ a partir de uma população $N(\theta,\theta)$, onde $\theta>0$. Utilize a quantidade pivotal para definir um intervalo de confiança $1-\alpha$ para $\theta$.

%Casela 9.13
\item Seja $X$ uma única observação obtida da distribuição Beta$(\theta,1)$.

\begin{enumerate}[a)]
\item Assuma $Y = -(\log(X))^{-1}$. Calcule o coeficiente de confiança do intervalo $[1/(2Y), 1/Y]$ para $\theta$. 
\item Mostre que $X^{\theta}$ é uma quantidade pivotal.
\item Construa um intervalo de confiança utilizando a quantidade pivotal $X^{\theta}$. Este intervalo deve
possuir um coeficiente de confiança igual a 0.95.
\end{enumerate}



% Casela 9.2 (quase)
\item 
Seja $X_1, \ldots, X_n$ uma amostra aleatória da $N(\theta; 1)$. Encontre uma quantidade pivotal para este problema e a utilize para obter um intervalo de confiança para $\theta$ com coeficiente de confiança igual a 0.95.


\item 
Seja $X_1, \ldots, X_n$ uma amostra aleatória de tamanho $n = 10$ da Exponencial$(\lambda); \lambda > 0$.

\begin{enumerate}[a)]
\item Mostre que $\lambda\sum X_i$ é uma quantidade pivotal.
\item Construa um intervalo de confiança utilizando a quantidade pivotal do item anterior. Este intervalo deve possuir um coeficiente de confiança igual a 0.90.
\item Mostre que $2\lambda\sum X_i$ também é uma quantidade pivotal.
\end{enumerate}


\item Suponha que a proporção $p$ de itens defeituosos, em uma grande população de itens, seja desconhecida. Deseja-se testar as seguintes hipóteses $H_0 : p = 0,2$ versus $H_1 : p \neq 0,2$. Considere que uma amostra aleatória de 20 itens seja retirada desta população e denote $Y$ = número de itens defeituosos na amostra. O seguinte procedimento de teste será usado: Rejeitar $H_0$ se $Y \geq 7$ ou $Y \leq 1$.

\begin{enumerate}[a)]
\item Determine a funcão poder deste teste.
 
 \item Calcule o valor da função poder para os seguintes pontos $p = \{0, 0.1, 0.2, 0.3, 0.4, 0.5, 0.6, 0.7, 0.8, 0.9, 1\}$. Faça o gráfico.
 
 \item Determine o tamanho do teste, ou seja, o valor de $\alpha = \sup_{\theta \in\Theta_0} \beta(\theta)$.
\end{enumerate}

\item Seja $X_1, \ldots, X_{10} $ uma amostra aleatória de tamanho $n = 10$ tal que $X_i \sim Bernoulli(\theta)$ onde $P(X_i = 1) = \theta = 1 - P(X_i = 0)$. Considere as hipóteses $H_0 : \theta \leq 1/2$ contra $H_1 : \theta > 1/2$. Assuma a seguinte regra de teste: Rejeitar $H_0$ se $\sum X_i \geq 6$.


\begin{enumerate}[a)]
 \item Determine a função poder do teste.
 \item Calcule a função poder para os seguintes pontos $p = \{0, 0.1, 0.2, 0.3, 0.4, 0.5, 0.6, 0.7, 0.8, 0.9, 1\}$. Faça o gráfico.
 \item Determine o tamanho do teste, ou seja, o valor de $\alpha = \sup_{\theta \in\Theta_0} \beta(\theta)$.
\end{enumerate}

\item Considere a variável aleatória $X$ com a seguinte densidade $f(x) = \theta x^{\theta-1}I_{(0,1)}(x)$. Para testar as hipóteses $H_0 : \theta \leq 1$ versus $H_1: \theta > 1$, uma única observação $(X_1)$ foi amostrada e o seguinte critério de rejeição foi adotado: rejeitar $H_0$ se $X_1 > 1/2$.

\begin{enumerate}[a)]
 \item Encontre a função poder deste teste.
 \item Determine o tamanho do teste.
\end{enumerate}



\end{enumerate}



\end{document}
% ----------------------------------------------------------------
