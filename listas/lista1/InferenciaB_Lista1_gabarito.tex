\documentclass[letter,11pt]{article}

\usepackage{amsfonts}
\usepackage{amsmath}
\usepackage{amssymb}
\usepackage[brazilian]{babel}
\usepackage{enumerate}
\usepackage[T1]{fontenc}
%\usepackage[ansinew,latin1]{inputenc}
\usepackage[utf8x]{inputenc}
\usepackage{multicol}
\setlength\columnseprule{0.5pt}

\newtheorem{exer}{Exercício}
\newtheorem{teo}{Teorema}

\newcommand{\var}{Var}
\newcommand{\E}{\mathbb{E}}

\newcommand{\mat}[1]{\mbox{\boldmath{$#1$}}}

\usepackage[letterpaper,top=3cm, bottom=2cm, left=2.5cm, right=2.5cm]{geometry}

\begin{document}

%\thispagestyle{empty}
\begin{center}{ \Large MAT02026 - Inferência B }\end{center}

\begin{center}
{\large  \sc Gabarito Lista 1 - Intervalos de Confiança e Intervalos de Credibilidade}
\end{center}
\vspace{5mm}

%%%%%%%%%%%%%%%%%%%%%%%%%%%%%%%%%%%%%%%%%%%%%%%%%%%%%%%%%%%%%%%%%%%%%%%%%%%%%%%
%Ex1
\begin{exer} \rm
% Se $L(X)$ e $U(X)$ satisfazem $P_\theta(L(X)\leqslant\theta)=1-\alpha_1$, 
% $P_\theta(U(X)\geqslant\theta)=1-\alpha_2$ e $L(x) \leqslant U(x)$ para todo 
% $x$, mostre que
% \[P_\theta(L(X)\leqslant\theta\leqslant U(X))=1-\alpha_1-\alpha_2.\]
\end{exer}
Defina $A=\{L(x)\leq \theta\}$ e $B=\{U(x)\geq\theta\}$.
Verifique que $\Prob(A\cup B)=1$ e $\Prob(A\cap B)=\Prob(L(x)\leq\theta\leq U(x)).$ 


%%%%%%%%%%%%%%%%%%%%%%%%%%%%%%%%%%%%%%%%%%%%%%%%%%%%%%%%%%%%%%%%%%%%%%%%%%%%%%%
% Ex2
\begin{exer} \rm
% Seja $X_1,\cdots,X_n$ uma a.a. com função de distribuição comum
% \[P(X_i\leqslant x)=\left\{
%                           \begin{array}{ll}
%                             0, & \hbox{se }x\leqslant0, \\
%                             \left(\frac{x}{\beta}\right)^\alpha, & \hbox{se }0<x<\beta, \\
%                             1, & \hbox{se }x\geqslant\beta.
%                           \end{array}
%                         \right.
% \]
% \begin{enumerate}[a)]
%   \item Se $\alpha$ é uma constante conhecida, $\alpha_0$, encontre um limite de confiança superior para $\beta$ com coeficiente de confiança de $95\%$.
%   \item Utilize os dados
% %\begin{center}
% \begin{tabular}{cccccccccccccc}
%  22,0 & 23,9 & 20,9 & 23,8 & 25,0 & 24,0 & 21,7 & 23,8 & 22,8 & 23,1 & 23,1 & 23,5 & 23,0 & 23,0 \\
% \end{tabular}
% %\end{center}
% \noindent para construir uma estimativa de intervalo para $\beta$. Assuma que $\alpha$ é conhecido e igual a estimativa gerada pelo seu estimador de máxima verossimilhança utilizando os dados acima. Interprete.
% \end{enumerate}
\end{exer}

\begin{enumerate}[a)]
\item % 9.3 do Casela
% o EMV está no exercicio 7.10 do Casela
Use que o EMV para $\beta$ é $X_{(n)}=max(X_i)$ . Como sabemos a distribuição de cada $X_i$, sabemos também a do max$(X_i)$. Assim, $X_{(n)}/\beta$ é uma quantidade pivotal e 
\[0.05=\Prob\left(\frac{X_{(n)}}{\beta}\leq q\right)
=\Prob\left(X_{(n)}\leq q\beta\right)=\left(\frac{q\beta}{\beta}\right)^{\alpha_0n}=q^{\alpha_0n}\] 

Assim, $q=0.05^{-\alpha_0n}$ e  $0.95=\Prob\left(\frac{X_{(n)}}{\beta}> q\right)=\Prob\left(\frac{X_{(n)}}{q}> \beta\right)$. Portanto 

$[\frac{X_{(n)}}{0.05^{-\alpha_0n}}]$ é a cota superior desejada.
\item Use $\hat{\alpha}=\frac{1}{n}\left[\sum_i\left( \log(X_{(n)}-\log(X_i)\right)\right]^{-1}$. $\hat{\alpha}= 12.59$ e  $X_{(n)} = 25$. Assim o IC fica $(0, 25.43)$.
\end{enumerate}

%%%%%%%%%%%%%%%%%%%%%%%%%%%%%%%%%%%%%%%%%%%%%%%%%%%%%%%%%%%%%%%%%%%%%%%%%%%%%%%
%Ex3
\begin{exer} \rm % Casela 9.17
% Seja $X_1,\cdots,X_{n}$ uma a.a. onde cada uma das variáveis aleatórias possui função densidade de probabilidade dada abaixo. Encontre um intervalo de confiança para $\theta$ com coeficiente de confiança $1-\alpha$.
% \begin{enumerate}[a)]
%   \item $f(x)=1$, para $\theta-\frac{1}{2}<x<\theta+\frac{1}{2}$.
%   \item $f(x)=\frac{2x}{\theta^2}$, para $0<x<\theta$, onde $\theta>0$.
% \end{enumerate}
\end{exer}

\begin{enumerate}[a)]
\item  $X-\theta\sim U[-1/2,1/2]$. Assim $\Prob(q_1<X-\theta<q_2)=q_2-q_1$. Logo escolhendo $q_2=1/2-\frac{\alpha}{2}$ e $q_1=-1/2+\frac{\alpha}{2}$ segue que o IC dado por  $[X-1/2+\frac{\alpha}{2};X+1/2-\frac{\alpha}{2}]$ tem $(1-\alpha)$100\%  de confiança para $\theta$. 
\item Observe que $X/\theta$ tem densidade $f(t)=2t$, $0\leq t\leq 1$. Assim  $\Prob(q_1<X/\theta<q_2)=q_2^2-q_1^2$. Escolha $q_1=\sqrt{\alpha/2}$ e $q_2=\sqrt{1-\alpha/2}$.
\end{enumerate}


%%%%%%%%%%%%%%%%%%%%%%%%%%%%%%%%%%%%%%%%%%%%%%%%%%%%%%%%%%%%%%%%%%%%%%%%%%%%%%%
% Ex4 - Casela 9.12
\begin{exer} \rm
% Encontre uma quantidade pivotal com base em uma amostra aleatória de tamanho $n$ a partir de uma população $N(\theta,\theta)$, onde $\theta>0$. Utilize a quantidade pivotal para definir um intervalo de confiança $1-\alpha$ para $\theta$.
\end{exer}

Use $\frac{\overline{X}-\theta}{S/\sqrt{n}}$ como pivô. (É possível usar outro pivô?)


%%%%%%%%%%%%%%%%%%%%%%%%%%%%%%%%%%%%%%%%%%%%%%%%%%%%%%%%%%%%%%%%%%%%%%%%%%%%%%%
% Ex5 - Casela 9.13
\begin{exer} \rm
% Seja $X$ uma única observação obtida da distribuição Beta$(\theta,1)$.
% \begin{enumerate}[a)]
%   \item Assuma $Y = -(\log(X))^{-1}$. Calcule o coeficiente de confiança do intervalo $[1/(2Y), 1/Y]$ para $\theta$. 
%   \item Mostre que $X^{\theta}$ é uma quantidade pivotal.
%   \item Construa um intervalo de confiança utilizando a quantidade pivotal $X^{\theta}$. Este intervalo deve
% possuir um coeficiente de confiança igual a 0.95.
% \end{enumerate}
\end{exer}

\begin{enumerate}[a)]
\item $Y=-((log(X))^{-1}$, $f_Y(y)=\frac{\theta}{y^2}e^{-\theta/y}$, $0<y<\infty$. Assim $P(1/2Y\leq \theta\leq 1/Y) = P(1/2\theta \leq Y \leq 1/\theta) = \int_{1/2\theta}^{1/\theta}f_Y(y)dy=0.2325$
\item $X^\theta \sim U(0,1)$. Assim, $\Prob(q_1\leq X^\theta\leq q_2)=q_2-q_1$. 
\item $\left[\frac{\log(0.975)}{\log X} ; \frac{\log(0,025)}{\log X}\right]$.
\end{enumerate}


%%%%%%%%%%%%%%%%%%%%%%%%%%%%%%%%%%%%%%%%%%%%%%%%%%%%%%%%%%%%%%%%%%%%%%%%%%%%%%%
% Ex6
% Casela 9.2 (quase)
\begin{exer} \rm
% Seja $X_1, \ldots, X_n$ uma amostra aleatória da $N(\theta; 1)$. Encontre uma quantidade pivotal para este problema e a utilize para obter um intervalo de confiança para $\theta$ com coeficiente de confiança igual a 0.95.
\end{exer}

$\left[\overline{X}-1,96\sqrt{1/n};  \overline{X}+1,96\sqrt{1/n}\right]$

\newpage
%%%%%%%%%%%%%%%%%%%%%%%%%%%%%%%%%%%%%%%%%%%%%%%%%%%%%%%%%%%%%%%%%%%%%%%%%%%%%%%
% Ex7
\begin{exer} \rm
% Seja $X_1, \ldots, X_n$ uma amostra aleatória de tamanho $n = 10$ da $Exponencial(\lambda)$, $\lambda > 0$.
% \begin{enumerate}[a)]
%   \item Mostre que $\lambda \sum_{i=1}^n X_i$ é uma quantidade pivotal.
%   \item Construa um intervalo de confiança utilizando a quantidade pivotal do item anterior. Este intervalo deve possuir um coeficiente de confiança igual a 0.90.
%   \item Mostre que $2\lambda \sum_{i=1}^n X_i$ também é uma quantidade pivotal.
% \end{enumerate}
\end{exer}

\begin{enumerate}[a)]
\item $\lambda\sum_{i=1}^{10} X_i \sim Gama(10,1)$.

\item $\left[\frac{5,425}{\sum_{i=1}^{10} X_i} ; \frac{15,705}{\sum_{i=1}^{10} X_i}\right]$

\item $2\lambda\sum_{i=1}^{10} X_i \sim Gama(10,1/2)$.
\end{enumerate}


%%%%%%%%%%%%%%%%%%%%%%%%%%%%%%%%%%%%%%%%%%%%%%%%%%%%%%%%%%%%%%%%%%%%%%%%%%%%%%%
% Ex8
% asintotico - exemplo 5.4.2. Bolfarine e Sandoval
\begin{exer} \rm
% Seja $X_1, \ldots, X_n$ uma amostra aleatória de tamanho $n$ da variável aleatória $Exponencial(\lambda)$, $\lambda > 0$. 
% \begin{enumerate}[a)]
%   \item Encontre um IC assintótico para $\lambda$ baseado na distribuição do EMV $\hat \lambda_{EMV}$.
%   \item Compare com o IC obtido no ítem (b) do Exercício 7, considerando um "$n$ grande".
% \end{enumerate}
\end{exer}
Exemplo 5.4.2 do livro Bolfarine e Sandoval

%%%%%%%%%%%%%%%%%%%%%%%%%%%%%%%%%%%%%%%%%%%%%%%%%%%%%%%%%%%%%%%%%%%%%%%%%%%%%%%
% Ex9
\begin{exer} \rm
% Seja $X_1, \ldots, X_n$ uma amostra aleatória de tamanho $n$ da variável aleatória $Poisson(\lambda)$, $\lambda > 0$. 
% \begin{enumerate}[a)]
%   \item Encontre um IC assintótico para $\lambda$ baseado na distribuição do EMV $\hat \lambda_{EMV}$.
%   \item Encontre um IC assintótico para $g(\lambda) = \frac{\lambda}{1-\lambda}$.
% \end{enumerate}
\end{exer}

\begin{enumerate}[a)]
   \item Com base na teoria de verossimilhança, quando $n \rightarrow \infty$ 
   temos que (eficiência assintótica) 
   $$ \sqrt n (\hat \lambda - \lambda) \rightarrow Normal \left( 0, I^{-1}(\lambda) \right).$$
   Para $\hat \lambda = \overline X = \sum_{i=1}^n X_i$ e $I^{-1}(\lambda) = 
   nI^{-1}_1(\lambda) = \frac{n}{\lambda}$. Então uma quantidade pivotal 
   assintótica é dada por 
   $$ Q(\boldsymbol{X}, \lambda) = \frac{\overline X - \lambda}{\sqrt{\frac{\lambda}{n}}} \sim Normal(0,1).$$
   
    \item Para $g(\hat{\theta})=\frac{\hat{\theta}}{1-\hat{\theta}}$, segue da 
    simples aplicação do método Delta.
    $$\sqrt n \left( g(\hat \lambda) - g(\lambda) \right) \rightarrow Normal \left( 0, \frac{(g'(\lambda))^2}{I(\lambda)} \right), \text{ quando } n \rightarrow \infty.$$
\end{enumerate}

%%%%%%%%%%%%%%%%%%%%%%%%%%%%%%%%%%%%%%%%%%%%%%%%%%%%%%%%%%%%%%%%%%%%%%%%%%%%%%%
% Ex10
\begin{exer} \rm
\end{exer}

% Resolver os exercícios 9.26  e 9.29 (a) do livro Casella e Berger.
\begin{itemize}
  \item 9.26 do livro Casella e Berger.
  \begin{enumerate}[a)]
    \item Função de verossimilhança: Se $\boldsymbol{X} = (X_1, \ldots, X_n)$ é 
    uma amostra aleatória de $X \sim Beta(\theta, 1)$, então para $\boldsymbol{X} 
    = \boldsymbol{x}$ temos que 
    
    $$ L(\theta) = f(\boldsymbol{x}; \theta) = \prod_{i=1}^n \frac{\Gamma(\theta + 1)}
    {\Gamma(\theta) \Gamma(1)} x_i^{\theta-1} (1-x_i)^{1-1}I_{[0,1]}(x_i) = 
    \theta^n \prod_{i=1}^n x_i^{\theta-1}.$$ \\
    
    Obs. 1: Se $\theta$ for inteiro é fácil mostrar que $\frac{\Gamma(\theta + 1)}
    {\Gamma(\theta) \Gamma(1)} = \theta$, e para $\theta$ real? \\
    
    Obs. 2: Note que $\prod_{i=1}^n x_i^{\theta-1} = e^{ \log \prod_{i=1}^n 
    x_i^{\theta-1} } = e^{ \sum_{i=1}^n (\theta-1) \log x_i }.$

    \item Distribuição \textit{a priori}: Se $\theta \sim Gama(r, \lambda)$ então
    $$ \pi(\theta) = \frac{\lambda^r}{\Gamma(r)} \theta^{r-1} e^{-\lambda \theta}.$$
    
    \item Distribuição \textit{a posteriori}: Usando o princípio da proporcionalidade 
    \begin{eqnarray}
    \pi(\theta \vert \boldsymbol{x}) 
        & \propto & \theta^n \: e^{ \sum_{i=1}^n (\theta-1) \log x_i } \left[ 
          \frac{\lambda^r}{\Gamma(r)} \right] \theta^{r-1} \: e^{-\lambda \theta} \nonumber \\
        & \propto & \theta^{n} \: e^{ \theta \sum_{i=1}^n \log x_i } \theta^{r-1} \:
          e^{-\lambda \theta} \nonumber \\
        & = & \theta^{n+r-1} \: e^{-(\lambda - \sum_{i=1}^n \log x_i) \theta}. \nonumber
    \end{eqnarray}
    Note que o núcleo da distribuição \textit{a posteriori} tem a forma de uma distribuição 
    $Gama$ tal que 
    $$ \theta \vert \boldsymbol{x} \sim Gama(n+r, \lambda - \sum_{i=1}^n \log x_i).$$
    
    \item Por fim, precisamos encontrar $[t_1, t_2]$ tal que $\int_{t_1}^{t_2} \pi(\theta \vert \boldsymbol{x}) d\theta = 1-\alpha$.
    
    
  \end{enumerate}
  
  \item 9.29 (a) do livro Casella e Berger
  Ver exemplo 7.2.14 do mesmo livro para distribuição \textit{a posteriori}, então mostrar como obter o intervalo a partir da distribuição.
\end{itemize}


% % Ex. 9.29 Casella - ICredibilidade
% 
% 
% % \begin{exer} \rm
% % Suponha que a proporção $p$ de itens defeituosos, em uma grande população de itens, 
% % seja desconhecida. Deseja-se testar as seguintes hipóteses $H_0 : p = 0,2$ versus 
% % $H_1 : p \neq 0,2$. Considere que uma amostra aleatória de 20 itens seja retirada 
% % desta população e denote $Y$ = número de itens defeituosos na amostra. O seguinte 
% % procedimento de teste será usado: Rejeitar $H_0$ se $Y \geq 7$ ou $Y \leq 1$.
% % \begin{enumerate}[a)]
% %   \item Determine a funcão poder deste teste.
% %   \item Calcule o valor da função poder para os seguintes pontos 
% %   $p = \{0, 0.1, 0.2, 0.3, 0.4, 0.5, 0.6, 0.7, 0.8, 0.9, 1\}$. Faça o gráfico.
% %   \item Determine o tamanho do teste, ou seja, o valor de $\alpha = \sup_{\theta 
% %   \in\Theta_0} \beta(\theta)$.
% % \end{enumerate}
% % \end{exer}
% % 
% % 
% % \begin{exer} \rm
% % Seja $X_1, \ldots, X_{10} $ uma amostra aleatória de tamanho $n = 10$ tal que 
% % $X_i \sim Bernoulli(\theta)$ onde $P(X_i = 1) = \theta = 1 - P(X_i = 0)$. 
% % Considere as hipóteses $H_0 : \theta \leq 1/2$ contra $H_1 : \theta > 1/2$. 
% % Assuma a seguinte regra de teste: Rejeitar $H_0$ se $\sum X_i \geq 6$.
% % \begin{enumerate}[a)]
% %   \item Determine a função poder do teste.
% %   \item Calcule a função poder para os seguintes pontos $p = \{0, 0.1, 0.2, 0.3, 0.4, 0.5, 0.6, 0.7, 0.8, 0.9, 1\}$. Faça o gráfico.
% %   \item Determine o tamanho do teste, ou seja, o valor de $\alpha = \sup_{\theta \in\Theta_0} \beta(\theta)$.
% % \end{enumerate}
% % \end{exer}
% % 
% % 
% % \begin{exer} \rm
% % Considere a variável aleatória $X$ com a seguinte densidade $f(x) = \theta x^{\theta-1}I_{(0,1)}(x)$. Para testar as hipóteses $H_0 : \theta \leq 1$ versus $H_1: \theta > 1$, uma única observação $(X_1)$ foi amostrada e o seguinte critério de rejeição foi adotado: rejeitar $H_0$ se $X_1 > 1/2$.
% % \begin{enumerate}[a)]
% %   \item Encontre a função poder deste teste.
% %   \item Determine o tamanho do teste.
% % \end{enumerate}
% % \end{exer}

\end{document}