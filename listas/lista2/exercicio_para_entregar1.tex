% ----------------------------------------------------------------
% AMS-LaTeX Paper ************************************************
% **** ---------------------------------------------------------
\documentclass[11pt,addpoints]{exam}
\usepackage{geometry}
\geometry{verbose,tmargin=1.5cm,bmargin=1.2cm,lmargin=1cm,rmargin=1.5cm}
\usepackage[T1]{fontenc}
\usepackage[utf8]{inputenc}
%\usepackage{babel}
\usepackage{amsmath}
\usepackage{amsfonts}
\usepackage{mathtools}
\usepackage[usenames,dvipsnames]{pstricks}
\usepackage{amssymb}
\usepackage{amsthm}

% Para redefinir o texto ao lado das pontuações:
\pointpoints{ponto}{pontos}


\usepackage{graphicx}

\usepackage{booktabs}

\usepackage{caption}

\usepackage{enumerate}

\setlength{\parindent}{0pt}

\usepackage{mleftright} % corrects spacing after \left( and before \right), etc
\mleftright

\newcommand{\E}{\mathbb{E}}
\newcommand{\Prob}{\mathbb{P}}
\newcommand{\T}{\mathbb{T}}
\newcommand{\prob}{\mathbb{P}}

\begin{document}
% Título do Artigo
\title{Exercício para entregar dia 03/05}

% Definição do(s) autor(es). Observe que é possível colocar
% uma nota de agradecimento, ou algo semelhante.

\author{
  Prof. Marcio Valk \\
  Disciplina: Inferência B\\
  %\and
  %Author 4 and Author 5  \\
  %Institution B
 % \and
  %Author 5  \\
  %Institution C
  \date{}
}

% Data. Se você quiser, pode entrar com o texto desejado no campo \date
% Caso queira a data do dia da compilação, exclua o comando \date
% Caso não queira que nada seja impresso no lugar da data, use \date{}

\maketitle


\begin{enumerate}[1.]

\item Seja $X\in\{1,2,3,4\}$ uma variável aleatória com função massa de probabilidade $P_\theta(X=k)$, para $\theta\in\Theta=\{0,1\}$ e $k\in\{1,2,3,4\}$ dada pela seguinte tabela

\begin{center}
\begin{tabular}{c|cccc}
&$P_\theta(X=1)$&$P_\theta(X=2)$&$P_\theta(X=3)$&$P_\theta(X=4)$\\
\hline
$\theta=0$ &0.02&0.02&0.03&0.93\\
$\theta=1$ &0.10&0.20&0.30&0.40\\
\end{tabular}
\end{center}
Considere as hipóteses $H_0:\theta=0$ vs. $H_1:\theta=1$. O teste A rejeita $H_0$ se $X\leq 2$ e o teste B rejeita $H_0$ se $X$ é par. 
 Use o Lema de Neyman-Pearson para encontrar o teste MP para um nível de significancia de 5\%.


\end{enumerate}

\end{document}
% ----------------------------------------------------------------
