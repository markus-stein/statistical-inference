% ----------------------------------------------------------------
% AMS-LaTeX Paper ************************************************
% **** ---------------------------------------------------------
\documentclass[10pt,brazil,addpoints]{exam}
\usepackage{geometry}
\geometry{verbose,tmargin=1.5cm,bmargin=1.2cm,lmargin=1cm,rmargin=1.5cm}
\usepackage[T1]{fontenc}
\usepackage[utf8]{inputenc}
\usepackage{babel}
\usepackage{amsmath}
\usepackage{amsfonts}
\usepackage{mathtools}
\usepackage{breqn}
\usepackage[inline]{enumitem}

\usepackage[usenames,dvipsnames]{pstricks}
%\usepackage{epsfig}
\usepackage{pst-grad} % For gradients
\usepackage{pst-plot} % For axes

\pagestyle{plain}

% Para redefinir o texto ao lado das pontuações:
\pointpoints{ponto}{pontos}
% Para redefinir as margens:
%\extrawidth{1in}
%\extraheadheight{-.5in}

\usepackage{graphicx}

\usepackage{booktabs}

\usepackage{caption}

\usepackage{enumerate}

\setlength{\parindent}{0pt}

\usepackage{mleftright} % corrects spacing after \left( and before \right), etc
\mleftright

\newcommand{\E}{\mathbb{E}}
\newcommand{\Prob}{\mathbb{P}}

\begin{document}
% Título do Artigo
\title{GABARITO Lista 2}

% Definição do(s) autor(es). Observe que é possível colocar
% uma nota de agradecimento, ou algo semelhante.

\author{
  Profª. Márcia Barbian \\
  Disciplina: Inferência B\\
  %\and
  %Author 4 and Author 5  \\
  %Institution B
 % \and
  %Author 5  \\
  %Institution C
  \date{}
}

% Data. Se você quiser, pode entrar com o texto desejado no campo \date
% Caso queira a data do dia da compilação, exclua o comando \date
% Caso não queira que nada seja impresso no lugar da data, use \date{}

\maketitle


\begin{enumerate}[1.]

\item Seja $X_1, \ldots, X_n$ uma amostra aleatória da densidade $f(x/\theta)$ e $Q=q(X_1, \ldots, X_n; \theta)$ uma função de distribuição de $X_1, \ldots, X_n$ e $\theta$. A distribuição $Q$ será uma quantidade pivotal se não depender do parâmetro desconhecido $\theta$. Ex: $Q=\frac{\overline{X}-\theta}{\sigma/\sqrt{n}}\sim N(0,1)$.

\item Está errada, pois o parâmetro $\theta$ é fixo. A interpretação apropriada é: a probabilidade de que o intervalo aleatório $[0,3 ; 0,35]$ contenha o valor do parâmetro $\theta$ é de 0,95.

\item 
\begin{enumerate}[a)]
\item 

\begin{equation}
Q= \frac{\overline{X}-\theta} {\sigma/\sqrt{n}} \sim N(0,1)
\end{equation}

\item 

\begin{equation}
Q= \frac{\overline{X}-\theta} {S/\sqrt{n}} \sim t_{n-1}
\end{equation}

\item 

\begin{equation}
Q= \frac{(n-1)S^2}{\sigma^2} \sim \chi^2_{n-1}
\end{equation}

\end{enumerate}

\item IC para a média $\mu$ $[133.73 ; 136.26]$.

IC para a variância $\sigma^2$ $[3.35 ; 14.42]$.

\item O IC para a proporção $\theta$ é $[0.3834; 0.511]$. Há 99\% de probabilidade que o intervalo aleatório $[0.3834; 0.511]$ contenha a probabilidade de observar cara dessa moeda. Como o valor $0.5$ está contido no intervalo de confiança, então há evidências de que a moeda seja honesta.



\end{enumerate}




\end{document}
% ----------------------------------------------------------------
