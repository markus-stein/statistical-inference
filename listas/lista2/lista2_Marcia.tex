% ----------------------------------------------------------------
% AMS-LaTeX Paper ************************************************
% **** ---------------------------------------------------------
\documentclass[10pt,brazil,addpoints]{exam}
\usepackage{geometry}
\geometry{verbose,tmargin=1.5cm,bmargin=1.2cm,lmargin=1cm,rmargin=1.5cm}
\usepackage[T1]{fontenc}
\usepackage[utf8]{inputenc}
\usepackage{babel}
\usepackage{amsmath}
\usepackage{amsfonts}
\usepackage{mathtools}
\usepackage{breqn}
\usepackage[inline]{enumitem}

\usepackage[usenames,dvipsnames]{pstricks}
%\usepackage{epsfig}
\usepackage{pst-grad} % For gradients
\usepackage{pst-plot} % For axes

\pagestyle{plain}

% Para redefinir o texto ao lado das pontuações:
\pointpoints{ponto}{pontos}
% Para redefinir as margens:
%\extrawidth{1in}
%\extraheadheight{-.5in}

\usepackage{graphicx}

\usepackage{booktabs}

\usepackage{caption}

\usepackage{enumerate}

\setlength{\parindent}{0pt}

\usepackage{mleftright} % corrects spacing after \left( and before \right), etc
\mleftright

\newcommand{\E}{\mathbb{E}}
\newcommand{\Prob}{\mathbb{P}}

\begin{document}
% Título do Artigo
\title{Lista 2}

% Definição do(s) autor(es). Observe que é possível colocar
% uma nota de agradecimento, ou algo semelhante.

\author{
  Profª. Márcia Barbian \\
  Disciplina: Inferência B\\
  %\and
  %Author 4 and Author 5  \\
  %Institution B
 % \and
  %Author 5  \\
  %Institution C
  \date{}
}

% Data. Se você quiser, pode entrar com o texto desejado no campo \date
% Caso queira a data do dia da compilação, exclua o comando \date
% Caso não queira que nada seja impresso no lugar da data, use \date{}

\maketitle


\begin{enumerate}[1.]

\item O que é uma quantidade pivotal? Exemplifique.

\item A afirmação: ``Há 95\% de probabilidade do parâmetro $\theta$ estar contido no intervalo $[0,3 ; 0,35]$" está correta? Justifique.

\item Seja $X_1, \ldots, X_n \sim N(\mu, \sigma^2)$, indique uma quantidade pivotal para:

\begin{enumerate}[a)] 

\item $\mu$, com variância conhecida;

\item $\mu$, com variância desconhecida;

\item $\sigma^2$.

\end{enumerate}

\item Sabe-se que em indivíduos hipertensos a pressão distólica pode ser considerada com uma variável que apresenta distribuição normal com os parâmetros $\mu$ e $\sigma^2$ (ambos   desconhecidos). Uma amostra aleatória de 12 hipertensos é selecionada apresentando média de 135mmHg e s=2,4495mmHg. Encontre um intervalo com 90\% de confiança para a média populacional e para a variância populacional. Interprete os resultados.

\item Um experimento foi conduzido para verificar se uma moeda é honesta. O experimento consistiu em arremessar esta moeda e observar o resultado. Foram observadas 179 caras em uma amostra aleatória de 400 arremessos. Encontre um IC com 99\% para $\theta$ (probabilidade de cara). Interprete o resultado encontrado,  a moeda parece ser honesta?

\item Por que $H_1$ é chamada de hipótese de pesquisa?

\item Quais os tipos de erros de um teste de hipóteses? Utilize um exemplo e indique quais os erros possíveis que um pesquisador pode cometer ao fazer um teste de hipóteses.

\item Qual o comportamento da função poder ideal?

\item Explique com suas palavras qual a idéia do TRV.

\item Explique com suas palavras o que é um teste de hipóteses.

\item Explique com suas palavras o que é um intervalo de confiança.

\item Faça seguintes exercícios do livro Statistical Inference: 

\begin{enumerate}[a)] 

\item 8.1 

\item 8.2

\item 8.4

\item 8.5 (a) e (b)

\item 8.6 (a)

\item 8.7 (b)

\item 8.12

\item 8.13 (a) e (b)

\item 8.16

\item 8.17 (a) e (b)

\item 8.18

\item 8.23 (a)

\end{enumerate}




\end{enumerate}

\end{document}
% ----------------------------------------------------------------
