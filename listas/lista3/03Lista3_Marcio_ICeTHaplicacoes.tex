% ----------------------------------------------------------------
% AMS-LaTeX Paper ************************************************
% **** ---------------------------------------------------------
\documentclass[10pt,brazil,addpoints]{exam}
\usepackage{geometry}
\geometry{verbose,tmargin=1.5cm,bmargin=1.2cm,lmargin=1cm,rmargin=1.5cm}
\usepackage[T1]{fontenc}
\usepackage[utf8]{inputenc}
\usepackage{babel}
\usepackage{amsmath}
\usepackage{amsfonts}
\usepackage{mathtools}
\usepackage[usenames,dvipsnames]{pstricks}
\usepackage{amssymb}
\usepackage{amsthm}

% Para redefinir o texto ao lado das pontuações:
\pointpoints{ponto}{pontos}


\usepackage{graphicx}

\usepackage{booktabs}

\usepackage{caption}

\usepackage{enumerate}

\setlength{\parindent}{0pt}

\usepackage{mleftright} % corrects spacing after \left( and before \right), etc
\mleftright

\newcommand{\E}{\mathbb{E}}
\newcommand{\Prob}{\mathbb{P}}
\newcommand{\T}{\mathbb{T}}
\newcommand{\prob}{\mathbb{P}}

\begin{document}
% Título do Artigo
\title{Lista 3 (Não Entregar)}

% Definição do(s) autor(es). Observe que é possível colocar
% uma nota de agradecimento, ou algo semelhante.

\author{
  Prof. Marcio Valk \\
  Disciplina: Inferência B\\
  %\and
  %Author 4 and Author 5  \\
  %Institution B
 % \and
  %Author 5  \\
  %Institution C
  \date{}
}

% Data. Se você quiser, pode entrar com o texto desejado no campo \date
% Caso queira a data do dia da compilação, exclua o comando \date
% Caso não queira que nada seja impresso no lugar da data, use \date{}

\maketitle


\begin{enumerate}[1.]

\item Suponha que $p=30\%$ dos estudantes de uma escola sejam
mulheres. Colhemos uma amostra aleatória simples de $n=10$
estudantes e calculamos  $\hat{p}$=proporção de mulheres na amostra.
Qual a probabilidade de que $\hat{p}$ difira de $p$
em menos de 0.01?  E se $n=50$?
% Adaptado de:  Morettin & Bussab, Estatística Básica 5a edição, pág 276.
%https://www.ime.unicamp.br/~cnaber/Aula.p10.pdf

\medskip
\item Um engenheiro deseja testar a hipótese de que o percentual de peças defeituosas é inferior a 10\%. Uma amostra aleatória com 75 peças revelou 6 peças defeituosas.

\begin{enumerate}[a)]
\item Construa um intervalo de confiânça para a proporção com coeficiente de confiânça $\gamma=95$\%.
%https://www.ime.unicamp.br/~cnaber/Aula.p10.pdf
\end{enumerate}



\medskip
\item Suponha que estejamos interessados em estimar a porcentagem de
consumidores de um certo produto. Se a amostra de tamanho 300
forneceu 100 indivíduos que consomem o dado produto, determine:
\begin{enumerate}[a)]
\item O intervalo de confiança de p, com c.c.  de 95\%; interprete o
resultado.
\item O tamanho da amostra para que o erro da estimativa não
exceda 0.02 unidades com probabilidade de 95\%; interprete o
resultado.
\end{enumerate}
%https://www.ime.unicamp.br/~cnaber/Aula.p10.pdf




\medskip
\item Estão sendo estudados dois processos para conservar alimentos,
cuja principal variável de interesse  é o tempo de duração destes.
No processo A, o tempo $X$ de duração segue a distribuição
N($\mu_A$,100), e no processo B o tempo $Y$
obedece á distribuição N($\mu_B$,100).  Sorteiam-se duas amostras independentes:  a de A, com 16 latas, apresentou tempo médio de duração igual a 50, e a
de B, com 25 latas, duração média igual a 60.

\begin{enumerate}[a)]
\item Construa um IC para $\mu_A$ e $\mu_B$, separadamente.
\item Para verificar e os dois processos podem ter o mesmo desempenho, decidiu-se construir um IC para a diferença $\mu_A-\mu_B$.  Caso o zero pertença intervalo, pode-se concluir que existe evidência de igualdade dos processos.  Qual seria sua resposta?
\end{enumerate}




\medskip
\item Para a comparação(diferença) entre as médias de duas populações pode-se utilizar a uma quantidade pivotal que tem uma distribuição t-student, onde a variância utilizada é a variância combinada das duas amostras. Indique as pressuposições que deverão ser válidas para que essa metodologia seja adequada.



\medskip
\item Uma cadeia de lojas recebeu um novo modelo de aparelho som. Para determinar um processo adequado de promoção dos novos aparelhos, estudou-se a sua performance em termos de potência. O fabricante especificou que, em média, os aparelhos atingem 65 watts a 8 ohms. Obtida uma amostra de 8 aparelhos, verificou-se que a potência média foi de 63,1 watts com desvio padrão de 1,7 watts. Verifique se a informação do fabricante está condizente com os resultados amostrais, construindo um intervalo de confiânça com $\gamma=97$\%.



\medskip
\item Sabe-se que certa raça de bovinos em confinamento, alimentada com uma ração padrão, tem um aumento médio de peso igual a 60 kg durante os três primeiros meses de idade. Um lote de 10 novilhos, dessa mesma raça, recebeu um novo tipo de alimentação com novos concentrados. Mantendo-se as mesmas condições de manejo, os aumentos de peso foram
\begin{center}
\begin{tabular}{cccccccccc}
55 & 62 &  54 & 58 & 65 & 65 &  60 & 62 &  59 &  67
\end{tabular}
\end{center}
Fixando o nível de significância em 1\%, conclua sobre o novo tipo de alimentação. Estime a média por intervalo e relacione o resultado com o teste de hipótese.


\medskip
\item Diante de uma equipe de fiscais, a nutricionista responsável pelo cardápio de um restaurante declarou que o peso médio de uma determinada vitamina por bandeja de refeição é de 5,5 g. Foi retirada uma amostra de 25 bandejas do fornecimento diário de refeições desse restaurante, encontrando-se uma média de 5,2 g da vitamina e um desvio padrão de 1,2 g. Verifique a veracidade da informação da nutricionista, construindo um intervalo de confiânça com $\gamma=$5\%.


\medskip
\item Teste de hipótese é um procedimento estatístico destinado a verificar hipóteses relativas a parâmetros populacionais. Uma questão fundamental nesse processo é a taxa de erro de conclusão. Indique o motivo pelo qual poderão existir tais erros e quais são eles.




\medskip
\item O Instituto de Nutrição da América Central e Panamá fez um estudo intensivo de resultados de dietas publicados em revistas científicas. Uma dieta aplicada a 15 pessoas produziu os seguintes níveis de colesterol (em mg/l):
\begin{center}
\begin{tabular}{ccccccccccccccc}
204 & 108 &  140 &  152 &  158 &  129 &  175 &  146 &  157 &  174 &  192 &  194 &  144 & 152  & 135
\end{tabular}
\end{center}

Sabendo-se que o nível médio normal de colesterol é de 190 mg/l, verifique se a redução no teor médio de colesterol das pessoas submetidas a essa dieta foi significativa, com $\alpha=0,05$.



\medskip
\item Para testar a performance em termos de consumo de combustível de um novo carro compacto, o fabricante sorteou seis motoristas profissionais que dirigiram o automóvel de Pelotas a Porto Alegre. O consumo do carro (em litros) para cada um dos seis motoristas foi de
\begin{center}
\begin{tabular}{ccccccc}
27,2&     29,3&     31,5&     28,7&     30,2&  29,6
\end{tabular}
\end{center}

Baseado nesses dados e utilizando nível de significância de 5\%, o fabricante pode indicar que o consumo médio do novo carro é de 30 litros para viagens nesse percurso?



\medskip
\item Vinte observações de um tipo de matriz indicaram um tempo de vida média de 217 minutos desvio padrão de 20 minutos. Construa um intervalo de confiança, a 95\%, para a média da população de matrizes.





\medskip
\item Os dados a seguir representam o ganho obtido em um processo químico. Use $\alpha = 0,05$ e teste a hipótese de que nas condições atuais o ganho seja superior a 1,5.
\begin{center}
\begin{tabular}{cccccccccccccccc}
1,50 & 1,55 & 1,59 & 1,42 & 1,53 & 1,58 & 1,48 & 1,52 &
1,53 & 1,62 & 1,46 & 1,56 & 1,63 & 1,54 & 1,58 & 1,68
\end{tabular}
\end{center}



\medskip
\item Para investigar se o treinamento é ou não transferido pelo ácido nucléico, 10 ratos foram treinados em discriminar se havia luz ou escuridão. Posteriormente esses ratos foram mortos, o ácido nucléico extraído e injetados em 10 ratos. Simultaneamente o ácido nucléico de 10 ratos não treinados foram injetados em outros 10. Os 20 ratos foram observados durante um período e o número de erros relativos a cada rato está na tabela abaixo.
\begin{center}
\begin{tabular}{c|ccccccccccccccc}
Treinado & 7&9&6&11&13&8&7&13&12&9\\
\hline
Não treinado&12&8&9&13&14&9&8&10&7&15\\
\end{tabular}
\end{center}
Verifique se, em média, os ratos treinados erram tanto quanto os ratos não treinados. (Suponha  e use $\alpha= 0,05$).



\medskip
\item Um fabricante atesta que as máquinas de enchimento que ele produz apresentam um coeficiente de variação (CV) inferior a 2\%. Engenheiros da empresa desconfiam que o fabricante não está dizendo a verdade. Um experimento aleatório realizado com garrafas de 2 litros indicou $s^2=0,0024$ litros$^2$ para uma amostra de 15 garrafas. Teste essa hipótese para um nível de significância $\alpha = 0,05$.



\medskip
\item Para verificar o grau de adesão de uma nova cola para vidros, preparam-se dois tipos de montagem; cruzado (A), onde a cola é posta em forma de X, e quadrado (B), onde a fórmula é posta nas 4 bordas. O resultado para a resistência das duas amostras está abaixo. Para um nível de 5\% de significância que tipo de conclusão poderia ser tirada?
\begin{center}
\begin{tabular}{c|ccccccccccccccc}
Método A &16&14&19&18&19&20&15&18&17&18\\
\hline
Método B&13&19&14&17&21&24&10&14&13&15\\
\end{tabular}
\end{center}



\medskip
\item A fim de comparar a eficácia de dois operários, foram tomadas, para cada um, sete medidas do tempo gasto, em segundos, para realizar certa operação. Os resultados obtidos são dados a seguir. Pergunta-se se, ao nível de 5\% de significância, os operários devem ser considerados igualmente eficazes ou não.
\begin{center}
\begin{tabular}{c|ccccccccccccccc}
Operário A &35&32&40&36&35&32&33\\
\hline
Operário B &29&35&36&34&30&33&31\\
\end{tabular}
\end{center}




\medskip
\item A Testosterona é uma droga que tem sido ministrada a atletas com a intenção de aumentar a massa muscular. Um estudo foi conduzido com 22 atletas, onde 11 receberam uma determinada dose da droga, durante um período de seis semanas, e os outros 11 receberam um placebo. Ao final desse período foi medida a largura do músculo (em mm, determinados por raio X). Utilizando nível de significância igual a 5\%, responda aos itens abaixo.
\begin{center}
\begin{tabular}{c|ccccccccccccccc}
Indivíduos&1&2&3&4&5&6&7&8&9&10&11\\
\hline
Placebo &3,7&5,2&4,0&4,7&4,3&3,9&4,2&4,9&5,1&4,1&4,0\\
Droga &13,1&16,5&15,3&15,7&14,1&15,0&15,5&16,1&15,8&14,3&15,2\\
\end{tabular}
\end{center}


\begin{enumerate}[\bf (a)]
\item Verifique, através do teste F, se as variâncias das duas populações diferem entre si.
 \item Verifique se existe diferença significativa entre a largura média do músculo dos dois grupos.
\end{enumerate}


\medskip
\item
Cinco medidas do conteúdo de alcatrão em um cigarro $X$ acusaram: 14,5, 14,2, 14,4, 14,8, e 14,1 miligramas por cigarro. Este conjunto de cinco valores tem média 14,4 e desvio padrão 0,274. O leitor pretende testar a hipótese nula H0: $\mu = 14,1$ (conforme declarado no maço) ao nível de 0,05 de significância.

\begin{enumerate}[\bf (a)]
\item H0 seria aceita, contra a alternativa HA: $\mu \neq 14,1$?
\item H0 seria aceita, contra a alternativa HA: $\mu < 14,1$?
\item H0 seria aceita, contra a alternativa HA: $\mu > 14,1$?
\item Que suposições são necessárias para fazer o teste de hipóteses?
\end{enumerate}

\medskip
\item Suponha que um fabricante sem escrúpulos deseje uma ``prova científic'' de que um aditivo químico totalmente inócuo melhora o rendimento.

\begin{enumerate}[\bf (a)]
\item  Se um grupo de pesquisa analisa esse aditivo com um experimento, qual é a probabilidade de chegar a um ``resultado significativo'' com $\alpha = 0,05$ (para promover o aditivo com “afirmações científicas”) mesmo que o aditivo seja totalmente inócuo?

\item Se dois grupos independentes de pesquisa analisam o aditivo, qual é a probabilidade de que pelo menos um deles chegue a um ``resultado significativo'', mesmo que o aditivo seja totalmente inócuo?

\item Se 32 grupos independentes de pesquisa analisam o aditivo, qual é a probabilidade de que pelo menos um deles chegue a um ``resultado significativo'', mesmo que o aditivo seja totalmente inócuo?
\end{enumerate}


\medskip
\item Suponha que um farmacêutico pretenda achar um novo medicamento  para reduzir inchaço. Para tanto, ele fabrica 20 medicamentos diferentes e testa cada um deles, ao nível de 10\% de significância, quanto a finalidade em vista.

\begin{enumerate}[\bf (a)]
\item Qual a probabilidade de ao menos um deles ``se revelar'' eficaz mesmo que todos sejam totalmente inócuos?

\item Qual a probabilidade de mais de um deles ``se revelarem'' eficazes, mesmo que todos sejam totalmente inócuos?
\end{enumerate}





 
\end{enumerate}

\end{document}
% ----------------------------------------------------------------
