% ----------------------------------------------------------------
% AMS-LaTeX Paper ************************************************
% **** ---------------------------------------------------------
\documentclass[10pt,brazil,addpoints]{exam}
\usepackage{geometry}
\geometry{verbose,tmargin=1.5cm,bmargin=1.2cm,lmargin=1cm,rmargin=1.5cm}
\usepackage[T1]{fontenc}
\usepackage[utf8]{inputenc}
\usepackage{babel}
\usepackage{amsmath}
\usepackage{amsfonts}
\usepackage{mathtools}
\usepackage{breqn}
\usepackage[inline]{enumitem}

\usepackage[usenames,dvipsnames]{pstricks}
%\usepackage{epsfig}
\usepackage{pst-grad} % For gradients
\usepackage{pst-plot} % For axes

\pagestyle{plain}

% Para redefinir o texto ao lado das pontuações:
\pointpoints{ponto}{pontos}
% Para redefinir as margens:
%\extrawidth{1in}
%\extraheadheight{-.5in}

\usepackage{graphicx}

\usepackage{booktabs}

\usepackage{caption}

\usepackage{enumerate}

\setlength{\parindent}{0pt}

\usepackage{mleftright} % corrects spacing after \left( and before \right), etc
\mleftright

\newcommand{\E}{\mathbb{E}}
\newcommand{\Prob}{\mathbb{P}}

\begin{document}
% Título do Artigo
\title{Lista 4}

% Definição do(s) autor(es). Observe que é possível colocar
% uma nota de agradecimento, ou algo semelhante.

\author{
  Profª. Márcia Barbian \\
  Disciplina: Inferência B\\
  %\and
  %Author 4 and Author 5  \\
  %Institution B
 % \and
  %Author 5  \\
  %Institution C
  \date{}
}

% Data. Se você quiser, pode entrar com o texto desejado no campo \date
% Caso queira a data do dia da compilação, exclua o comando \date
% Caso não queira que nada seja impresso no lugar da data, use \date{}

\maketitle


\begin{enumerate}[1.]

\item Seja $X_1, \ldots, X_n$ uma amostra aleatória da variável $X \sim $ Geométrica$(\theta)$.

\begin{enumerate}[a)]
\item Encontre o TRV para as hipóteses $H_0: \theta=\theta_0$ contra $H_1: \theta \neq \theta_0$.

\item Encontre o teste uniformemente mais poderoso (UMP) para testar $H_0: \theta=\theta_0$ versus $H_1: \theta=\theta_1$, em que $\theta_0<\theta_1$ são especificados pelo pesquisador.

\item  Encontre o teste UMP para $H_0: \theta=0.3$ versus $H_1: \theta=0.5$.

\item Dado o teste UMP calculado na letra (c), considere que $n=5$ e $\alpha=0.04$, qual o critério de rejeição? Qual o erro tipo II? Se a amostra observada fosse $x=\{4,5,3,2,5\}$ qual a sua decisão?

\item Gere artificialmente, no software R, uma amostra de $n= 100$ de uma distribuição geométrica com parâmetro $\theta=0.5$.  Dado o teste UMP calculado na letra (b) e $\alpha=0.04$, qual o critério de rejeição? Qual o erro tipo II? Compare com o resultado encontrado na letra (d).

\end{enumerate}

\item Seja $X_1, \ldots, X_n$ uma amostra aleatória de tamanho $n$ da distribuição $N(\mu, 1)$.


\begin{enumerate}[a)]
\item Encontre o teste UMP para as hipóteses $H_0: \mu=0$ contra $H_1: \mu=1$.

\item Suponha que $n=9$ e $\alpha=0,05$. Qual é a região crítica do teste obtido em (a).

\item Faça o gráfico da função poder da letra (b).
\end{enumerate}


\item Seja $X_1, \ldots, X_n$ uma amostra aleatória da variável $X$ com a seguinte função densidade:

$$
f(x/\theta)=\theta x^{\theta-1} I_{(0,1)}, \mbox{ em que } \theta>0.
$$

\begin{enumerate}[a)]
\item O teste mais poderoso para $H_0: \theta=1$ versus $H_1: \theta=2$ rejeitará $H_0$ se $[\sum_{i=1}^n -\log (x_i) \leq a]$ onde $a$ é uma constante. Mostre este resultado.

\item Sendo $n=2$ e $\alpha=[1-\log(2)]/2$ qual seria a região crítica? Dica: se $X \sim$ Beta($\theta$, 1) então $-\log(X)\sim$ Exp($\theta$).
\end{enumerate}

\item Seja $X_1, \ldots, X_n$ uma amostra aleatória obtida da distribuição Poisson($\theta$). Encontre o teste UMP para as hipóteses $H_0: \theta=\theta_0$ versus $H_1: \theta=\theta_1$, considere que $\theta_0 < \theta_1$


\item Seja $X_1, \ldots, X_n$ uma amostra aleatória obtida da distribuição $N(0, \sigma^2)$.

\begin{enumerate}[a)]
\item Encontre o teste UMP para $H_0:\sigma^2=\sigma_0^2$ versus $H_1:\sigma^2=\sigma_1^2$. Considere que $\sigma_0^2< \sigma_1^2$.

\item Sendo $\sigma_0^2=1$, $\sigma^2_1=2$, $n=2$ e $\alpha=0.05$, qual seria a região crítica? 


\item Gere artificialmente, no software R, uma amostra de $n= 10$ da distribuição normal com parâmetros $(\mu=0, \sigma^2=2)$.  Dado o teste UMP calculado na letra (a), as hipóteses $\sigma_0^2=1$, $\sigma^2_1=2$ e $\alpha=0.04$, qual o critério de rejeição? Qual o erro tipo II? Qual a sua decisão dado essa amostra?

\item Refaça o item anterior, mas utilize uma amostra de tamanho $n=100$.  Compare com o resultado encontrado na letra (c).



\end{enumerate}

\item O que é p-valor?

\item O que é nível descritivo amostral? 
\item O que é nível de significância?

\end{enumerate}




\end{document}
% ----------------------------------------------------------------
