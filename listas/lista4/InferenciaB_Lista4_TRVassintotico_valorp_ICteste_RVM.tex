\documentclass[letter,11pt]{article}

\usepackage{amsfonts}
\usepackage{amsmath}
\usepackage{amssymb}
\usepackage[brazilian]{babel}
\usepackage{enumerate}
\usepackage[T1]{fontenc}
%\usepackage[ansinew,latin1]{inputenc}
\usepackage[utf8x]{inputenc}
\usepackage{multicol}
\setlength\columnseprule{0.5pt}

\newtheorem{exer}{Exercício}
\newtheorem{teo}{Teorema}

\newcommand{\var}{Var}
\newcommand{\E}{\mathbb{E}}

\newcommand{\mat}[1]{\mbox{\boldmath{$#1$}}}

\usepackage[letterpaper,top=3cm, bottom=2cm, left=2.5cm, right=2.5cm]{geometry}

\begin{document}

%\thispagestyle{empty}
\begin{center}{ \Large MAT02026 - Inferência B }\end{center}

\begin{center}
{\large  \sc Lista 4 - TRV assintótico. IC como testes, valor $p$ e RVM}
\end{center}
\vspace{5mm}

%%%%%%%%%%%%%%%%%%%%%%%%%%%%%%%%%%%%%%%%%%%%%%%%%%%%%%%%%%%%%%%%%%%%%%%%%%%%%%%
% Exercicio estendido prova 1
\begin{exer} \rm
Seja $X_1$ uma única observação obtida da distribuição $Beta(\theta, 1)$
\begin{enumerate}[a)]
  \item Mostre que $X_1^{\theta}$ é uma quantidade pivotal.
  \item Construa um intervalo com 95\% de confiança utilizando a quantidade 
  pivotal $X_1^{\theta}$.
  \item Assuma *a piori* $\theta \sim Gama(\alpha=1, \beta)$, encontre um 
  intervalo $1 - \alpha$ de credibilidade para $\theta$. Compare os intervalos.
  \item Comente sobre as suposições para construirmos intervalos segundo as 
  duas abordagens.
  \item Teste de hipóteses frequentistas e bayesianos também podem ser 
  construídos com base nos intervalos de confiança e de credibilidade, 
  respectivamente. Gere uma amostra de tamanho $n=1$ da distribuição 
  $Beta(1,5; 1)$ e teste a hipótese $H_0 : \theta = 1$ contra $H_1: \theta \neq 1$.
  \item Calcule o valor $p$ para os testes acima. Justifique os cálculos e interprete os valores $p$.
\end{enumerate}
\end{exer}


\begin{exer} \rm
Quiz sobre valor $p$.  
\begin{enumerate}[a)]
  \item Qual o significado do valor $p$ na prática? Como a ciência tem utilizado o valor $p$ para criar suas teorias? Cite exemplos. 
  \item Porque o uso do valor $p$ tem sido muito criticado recentemente? 
  \item Qual sua conclusão sobre o problema. Indique alternativas ao valor $p$.
\end{enumerate}
\end{exer}


\begin{exer} \rm
Seja $X_1, \ldots, X_n$ uma amostra aleatória de uma população com distribuição $N(\mu, \sigma^2)$

\begin{enumerate}[a)]

\item Se $\mu$ é desconhecido e $\sigma^2$, mostre que $Z=\sqrt{n}(\overline{X}-\mu_0)/\sigma$ é um teste de Wald para $H_0:\mu=\mu_0$.

\item Se $\sigma^2$ é desconhecido e $\mu$ é conhecido, encontre a Estaistica de Wald para testar $H_0:\sigma=\sigma_0$.
\end{enumerate}
\end{exer}


%questão 17 pag 259 do migon
\begin{exer} \rm
Seja $X_1, \ldots, X_n$ uma amostra aleatória da distribuição Exponencial $(\theta)$, suponha que queremos testar $H_0:\theta=1$.

\begin{enumerate}[a)]

\item Mostre que o TRV rejeita $H_0$ quando $\sum X_i<c$. 

\item Qual o valor de $c$ para $\alpha=0.05$.

\item Construa o teste assintótico da razão de verossimilhança e o teste de Wald e compare com o teste exato.

\item Gere aleatoriamente uma amostra de $n=20$ e $\theta=1.5$ de uma distribuição exponencial. Calcule os testes de Wald, verosmilhança assintótico e teste de verossimilhança exato para essa amostra. Repita o experimento 100 vezes e indique quantas vezes rejeita-se a $H_0$ a um nível de significância de 5\%. Compare os resultados.

\end{enumerate}
\end{exer}


%exercicio 12 wasserman pag 173
\begin{exer} \rm
Seja $X_1, \ldots, X_n$ uma amostra aleatória de uma distribuição de Poisson ($\lambda$).
  \begin{enumerate}[a)]
    \item Seja $\lambda_0>0$, encontre um teste de Wald de tamanho $\alpha$ para
$$
H_0: \lambda=\lambda_0  \mbox{ versus } H_1: \lambda \neq \lambda_0
$$

    \item Calcule o TRV para as hipóteses acima.

    \item Calcule o TRV assintótico para as hipóteses acima. 
  
    \item Calcule o Teste de Wald para as hipóteses acima.

    \item Seja $\lambda=1$ e $\lambda_0=0.8$, $n=20$ e $\alpha=0.05$. Simule $X_1, X_2, \ldots, X_n$ de uma distribuição de Poisson $(\lambda)$ e calcule os testes acima. Repita esse procedimento 100 vezes, calcule quantas vezes a hipótese nula foi rejeitada. Qual o valor do do erro tipo I para cada um dos testes?
\end{enumerate}
\end{exer}


\begin{exer} \rm
Verifique se as as distribuicoes dos exercícios 3, 4 e 5 possuem razão de verossimilhança monótona. 
\end{exer}


\begin{exer} \rm
Construa o teste uniformemente mais poderoso de tamanho $\alpha$ para:

  \begin{enumerate}
    \item Exercício 1 letra a, em que $H_0: \mu \leq \mu_0$.
    
    \item Exercício 2, em que $\theta \leq 1$.
    
    \item Exercício em que $\lambda \leq \lambda_0$.
  \end{enumerate}
\end{exer}

\end{document}