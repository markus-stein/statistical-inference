% ----------------------------------------------------------------
% AMS-LaTeX Paper ************************************************
% **** ---------------------------------------------------------
\documentclass[10pt,brazil,addpoints]{exam}
\usepackage{geometry}
\geometry{verbose,tmargin=1.5cm,bmargin=1.2cm,lmargin=1cm,rmargin=1.5cm}
\usepackage[T1]{fontenc}
\usepackage[utf8]{inputenc}
\usepackage{babel}
\usepackage{amsmath}
\usepackage{amsfonts}
\usepackage{mathtools}
\usepackage{breqn}
\usepackage[inline]{enumitem}

\usepackage[usenames,dvipsnames]{pstricks}
%\usepackage{epsfig}
\usepackage{pst-grad} % For gradients
\usepackage{pst-plot} % For axes

\pagestyle{plain}

% Para redefinir o texto ao lado das pontuações:
\pointpoints{ponto}{pontos}
% Para redefinir as margens:
%\extrawidth{1in}
%\extraheadheight{-.5in}

\usepackage{graphicx}

\usepackage{booktabs}

\usepackage{caption}

\usepackage{enumerate}

\setlength{\parindent}{0pt}

\usepackage{mleftright} % corrects spacing after \left( and before \right), etc
\mleftright

\newcommand{\E}{\mathbb{E}}
\newcommand{\Prob}{\mathbb{P}}

\begin{document}
% Título do Artigo
\title{GABARITO Lista 5}

% Definição do(s) autor(es). Observe que é possível colocar
% uma nota de agradecimento, ou algo semelhante.

\author{
  Profª. Márcia Barbian \\
  Disciplina: Inferência B\\
  %\and
  %Author 4 and Author 5  \\
  %Institution B
 % \and
  %Author 5  \\
  %Institution C
  \date{}
}

% Data. Se você quiser, pode entrar com o texto desejado no campo \date
% Caso queira a data do dia da compilação, exclua o comando \date
% Caso não queira que nada seja impresso no lugar da data, use \date{}

\maketitle


\begin{enumerate}[1.]

\item Seja a amostra aleatória $X_1, \ldots, X_n$ com distribuição Binomial Negativa$(r,\theta)$, com função de probabilidade:

$$p(x|r,\theta) = \binom {x+r-1} x \theta^r (1-\theta)^x$$

com $x = 0,1,2,\ldots$

\begin{enumerate}[a)]
\item A conjugada da binomial negativa é a distribuiçao beta.

Assumindo uma prior uniforme, beta(a,b), temos:

$$\theta \sim beta(a,b)$$

e a regra de atualização é dada por:

$$\theta | x \sim beta(a + rn, b + \sum_i x_i)$$

com $x = (x_1, x_2, \ldots, x_n)$.

\item Seja $\theta \sim beta(2,2)$.

Para $r=5, n=10, \sum_i x_i = 70$ temos $\theta|x \sim beta(52,72)$.

Sejam as hipóteses $H_0:\theta \leq 0.5$ vs. $H_1:\theta \gt 0.5$. 

Qual das hipóteses apresenta maior probabilidade à posteriori?

$$p(H_0|x) = p(\theta \leq 0.5|x) = pbeta(0.5,52,72) = 0.964$$

$$p(H_1|x) = 1 - p(H_0|x) = 0.036$$


\item $$O(H_1,H_0|x) = \frac{p(H_1|x)}{p(H_0|x)} = \frac{0.036}{0.964} = 0.037$$

Assumindo que as hipóteses tinham inicialmente igual probabilidade:

O fator de Bayes a favor de $H_1$:

$$B(x) = \frac{O(H_1,H_0|x)}{O(H_1,H_0)} = \frac{0.037}{1} = 0.037$$

Não existe evidência a favor de $H_1$.

\end{enumerate}

\item Seja $\theta$ a probabilidade de preferir o congelador de maior qualidade.

\begin{enumerate}[a)]
\item Modelo de dados proposto: $f(x|\theta) \sim binomial(\theta)$

Não se considera as diferenças de gostos individuais dos clientes.

\item $N=16,x=13$

Para $\theta_1 \sim beta(0.5,0.5)$, $\theta_1 \sim beta(13.5,3.5)$


$p(\theta_1 > 0.6 | x) = 1 - p(\theta_1 \leq 0.6 | x)$

Seja $H_0:\theta \geq 0.6$ vs. $H_1:\theta < 0.6$

Assumimos $p(H_0) = P(H_1) = 0.5$, logo $O(H_0,H_1) = 1$

$$B(x) = \frac{O(H_0,H_1|x)}{O(H_0,H_1)} = O(H_0,H_1|x) = \frac{p(\theta_1 \geq 0.6 | x)}{p(\theta_1 < 0.6 | x)} = \frac{0.9637}{0.036} = 26.6$$

Repetindo os cálculos para priors diferentes:

Para $\theta_2 \sim beta(1,1)$ obtemos $B(x) = 19$

Para $\theta_3 \sim beta(2,2)$ obtemos $B(x) = 13.3$

A prior aparenta ter um grande efeito no fator de Bayes.


\end{enumerate} 

\item $$\theta \sim Gamma(1,1)$$

$$X_i \sim Poisson(\theta)$$

Para $n=10, \sum_i x_i = 6$ temos:
$$\theta | x \sim Gamma(7,11)$$

Intervalo de confiança a 90%:


qgamma(0.05,7,11)

qgamma(0.95,7,11)


Região de credibilidade HPD a 90%:


library(TeachingDemos)

hpd(qgamma,conf=0.9, shape=7, rate=11)


\item prior: $\theta \sim beta(1,1)$

verosimilhança de $x$ casos em $n$ observações: $x \sim binomial(\theta)$

posterior: $\theta | x \sim beta(x+1,n-x+1)$

Temos as seguintes hipóteses:
+ $H_0: \theta = 0.2$
+ $H_1: \theta = 0.3$

Hipóteses à priori:
+ $p(H_0) = 0.25$
+ $p(H_1) = 0.75$

Tendo assim $O(H_0,H_1) = \frac{0.25}{0.75} = \frac{1}{3}$

Seja o fator de Bayes em favor de $H_0$: 

$$B(x) = \frac{O(H_0,H_1|x)}{O(H_0,H_1)}$$

Se o fator de bayes for maior que 1 (para favorecer $H_0$ como é pedido no enunciado). Assim:

$$B(x) = \frac{O(H_0,H_1|x)}{O(H_0,H_1)} \gt 1 \iff \frac{O(H_0,H_1|x)}{\frac{1}{3}} \gt 1 \iff O(H_0,H_1|x) \gt \frac{1}{3}$$

Ou seja,

$$O(H_0,H_1|x) \gt \frac{1}{3} \iff \frac{p(H_0|x)}{p(H_1|x)} \gt \frac{1}{3}\iff p(H_0|x) \gt 3p(H_1|x)$$

Ora, sabendo que os dados seguem uma binomial:

+ $p(H_0|x) = p(\theta = 0.2|x) = \binom n x 0.2^x 0.8^{n-x}$
+ $p(H_1|x) = p(\theta = 0.3|x) = \binom n x 0.3^x 0.7^{n-x}$

Juntando tudo:

$$\binom n x 0.2^x 0.8^{n-x} \gt 3 \binom n x 0.3^x 0.7^{n-x}$$

$$0.2^x 0.8^{n-x} \gt 3 \times 0.3^x 0.7^{n-x}$$

$$x \lt \frac{n - 8.67}{4.2}$$

\item

\item 

\begin{enumerate}[a)]
\item Parâmetros da priori $\alpha=4.8$ e 
$\beta=19.2 $. Logo, $\theta/x\sim Beta(30.8, 93.2)$

\item Estimativa MVG=0.244 e estimativa de bayes=0.248.

\item IC central =(0.17;0.32) e IC HPD =(0.174, 0.32)

\item $P(\theta \leq 0.5) \approx 1$
\end{enumerate}

\item 
\begin{enumerate}[a)]

\item IC HPD= (30.839 ; 33.079), 1º quartil=31,573, 3º quartil 32.344

\item ICP HPD=(30.868; 33.131), 1º quartil=31.610, 3º quartil= 32.389

\item IC clássico igual ao HPD do item b.



\end{enumerate}
\end{enumerate}




\end{document}
% ----------------------------------------------------------------
