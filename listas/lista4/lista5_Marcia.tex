% ----------------------------------------------------------------
% AMS-LaTeX Paper ************************************************
% **** ---------------------------------------------------------
\documentclass[10pt,brazil,addpoints]{exam}
\usepackage{geometry}
\geometry{verbose,tmargin=1.5cm,bmargin=1.2cm,lmargin=1cm,rmargin=1.5cm}
\usepackage[T1]{fontenc}
\usepackage[utf8]{inputenc}
\usepackage{babel}
\usepackage{amsmath}
\usepackage{amsfonts}
\usepackage{mathtools}
\usepackage{breqn}
\usepackage[inline]{enumitem}

\usepackage[usenames,dvipsnames]{pstricks}
%\usepackage{epsfig}
\usepackage{pst-grad} % For gradients
\usepackage{pst-plot} % For axes

\pagestyle{plain}

% Para redefinir o texto ao lado das pontuações:
\pointpoints{ponto}{pontos}
% Para redefinir as margens:
%\extrawidth{1in}
%\extraheadheight{-.5in}

\usepackage{graphicx}

\usepackage{booktabs}

\usepackage{caption}

\usepackage{enumerate}

\setlength{\parindent}{0pt}

\usepackage{mleftright} % corrects spacing after \left( and before \right), etc
\mleftright

\newcommand{\E}{\mathbb{E}}
\newcommand{\Prob}{\mathbb{P}}

\begin{document}
% Título do Artigo
\title{Lista 5}

% Definição do(s) autor(es). Observe que é possível colocar
% uma nota de agradecimento, ou algo semelhante.

\author{
  Profª. Márcia Barbian \\
  Disciplina: Inferência B\\
  %\and
  %Author 4 and Author 5  \\
  %Institution B
 % \and
  %Author 5  \\
  %Institution C
  \date{}
}

% Data. Se você quiser, pode entrar com o texto desejado no campo \date
% Caso queira a data do dia da compilação, exclua o comando \date
% Caso não queira que nada seja impresso no lugar da data, use \date{}

\maketitle


\begin{enumerate}[1.]

%Questão 10.35 (pag 513) do Casela

\item Seja $X_1, \ldots, X_n$ uma amostra aleatória de uma população com distribuição $N(\mu, \sigma^2)$

\begin{enumerate}[a)]

\item Se $\mu$ é desconhecido e $\sigma^2$, mostre que $Z=\sqrt{n}(\overline{X}-\mu_0)/\sigma$ é um teste de Wald para $H_0:\mu=\mu_0$.

\item Se $\sigma^2$ é desconhecido e $\mu$ é conhecido, encontre a Estaistica de Wald para testar $H_0:\sigma=\sigma_0$.


\end{enumerate}

%questão 17 pag 259 do migon
\item Seja $X_1, \ldots, X_n$ uma amostra aleatória da distribuição Exponencial $(\theta)$, suponha que queremos testar $H_0:\theta=1$.

\begin{enumerate}[a)]

\item Mostre que o TRV rejeita $H_0$ quando $\sum X_i<c$. 

\item Qual o valor de $c$ para $\alpha=0.05$.

\item Construa o teste assintótico da razão de verossimilhança e o teste de Wald e compare com o teste exato.

\item Gere aleatoriamente uma amostra de $n=20$ e $\theta=1.5$ de uma distribuição exponencial. Calcule os testes de Wald, verosmilhança assintótico e teste de verossimilhança exato para essa amostra. Repita o experimento 100 vezes e indique quantas vezes rejeita-se a $H_0$ a um nível de significância de 5\%. Compare os resultados.

\end{enumerate}


%exercicio 12 wasserman pag 173
\item Seja $X_1, \ldots, X_n$ uma amostra aleatória de uma distribuição de Poisson ($\lambda$).
\begin{enumerate}[a)]


\item Seja $\lambda_0>0$, encontre um teste de Wald de tamanho $\alpha$ para
$$
H_0: \lambda=\lambda_0  \mbox{ versus } H_1: \lambda \neq \lambda_0
$$

\item Calcule o TRV para as hipóteses acima.

\item Calcule o TRV assintótico para as hipóteses acima. 

\item Calcule o Teste de Wald para as hipóteses acima.

\item Seja $\lambda=1$ e $\lambda_0=0.8$, $n=20$ e $\alpha=0.05$. Simule $X_1, X_2, \ldots, X_n$ de uma distribuição de Poisson $(\lambda)$ e calcule os testes acima. Repita esse procedimento 100 vezes, calcule quantas vezes a hipótese nula foi rejeitada. Qual o valor do do erro tipo I para cada um dos testes?
\end{enumerate}

\item Verifique se as as distribuicoes dos exercícios 1,2 e 3 possuem razão de verossimilhança monótona. 

\item Construa o teste uniformemente mais poderoso de tamanho $\alpha$ para:

\begin{enumerate}
    \item Exercício 1 letra a, em que $H_0: \mu \leq \mu_0$.
    
    \item Exercício 2, em que $\theta \leq 1$.
    
    \item Exercício em que $\lambda \leq \lambda_0$.
\end{enumerate}



\end{enumerate}


\end{document}
% ----------------------------------------------------------------
