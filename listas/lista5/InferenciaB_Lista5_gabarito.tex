\documentclass[letter,11pt]{article}

\usepackage{amsfonts}
\usepackage{amsmath}
\usepackage{amssymb}
\usepackage[brazilian]{babel}
\usepackage{enumerate}
\usepackage[T1]{fontenc}
%\usepackage[ansinew,latin1]{inputenc}
\usepackage[utf8x]{inputenc}
\usepackage{multicol}
\setlength\columnseprule{0.5pt}

\newtheorem{exer}{Exercício}
\newtheorem{teo}{Teorema}

\newcommand{\var}{Var}
\newcommand{\E}{\mathbb{E}}

\newcommand{\mat}[1]{\mbox{\boldmath{$#1$}}}

\usepackage[letterpaper,top=3cm, bottom=2cm, left=2.5cm, right=2.5cm]{geometry}

\begin{document}

%\thispagestyle{empty}
\begin{center}{ \Large MAT02026 - Inferência B }\end{center}

\begin{center}
{\large  \sc Gabarito Lista 5 - TRV e IC para comparações de grupos}
\end{center}
\vspace{5mm}

%%%%%%%%%%%%%%%%%%%%%%%%%%%%%%%%%%%%%%%%%%%%%%%%%%%%%%%%%%%%%%%%%%%%%%%%%%%%%%%
\begin{exer} \rm
% Seja $\boldsymbol{X} = (X_1, \ldots, X_n)$ uma a. a. de $X \sim Normal(\mu_X, \sigma^2_X)$ e $\boldsymbol{Y} = (Y_1, \ldots, Y_m)$ uma a.a. de $Y \sim Normal(\mu_Y, \sigma^2_Y)$, tal que $\boldsymbol{X}$ e $\boldsymbol{Y}$ são independentes. Encontre o TRV para testar:
\begin{enumerate}[a)]
  \item %$H_0: \mu_X = \mu_Y$ contra $H_1: \mu_X \neq \mu_Y$ assumindo que $\sigma^2_X = \sigma^2_Y = \sigma^2$;
  Exemplo 6.5.6 do livro Bolfarine e Sandoval (2001). 
  
  \item %\textit{(Behrens-Fisher problem)} $H_0: \mu_X = \mu_Y$ contra $H_1: \mu_X \neq \mu_Y$ assumindo que $\sigma^2_X \neq \sigma^2_Y$;
  Exercício 8.42 do livro Casella e Berger (2002).
  
  
  \item %$H_0: \sigma^2_X = \sigma^2_Y$ contra $H_1: \sigma^2_X \neq \sigma^2_Y$.   
  Exemplo 6.5.7 do livro Bolfarine e Sandoval (2001).
\end{enumerate}
\end{exer}


\begin{exer} \rm
%\textit{(Teste $t$ pareado)} Seja $(X_1, Y_1), \ldots (X_n, Y_n)$ uma a.a. de $(X, Y) \sim Normal_2(\mu_X, \mu_Y, \sigma^2_X, \sigma^2_Y, \rho)$ e $\boldsymbol{Y} = (Y_1, \ldots, Y_m)$ uma a.a. de $Y \sim Normal(\mu_Y, \sigma^2_Y)$. Use o TRV para testar $H_0: \mu_X = \mu_Y$. Dica: mostre que $W_i = X_i - Y_i \sim Normal(\mu_W, \sigma^2_W)$. 
Exercício 8.39 do livro Casella e Berger (2002).
\end{exer}


\begin{exer} \rm
% Seja $\boldsymbol{X} = (X_1, \ldots, X_n)$ uma a.a. de $X \sim Bernoulli(\pi_1)$ e $\boldsymbol{Y} = (Y_1, \ldots, Y_m)$ uma a.a. de $Y \sim Bernoulli(\pi_2)$, tal que $\boldsymbol{X}$ e $\boldsymbol{Y}$ são independentes. Encontre o TRV para testar $H_0: \pi_1 = \pi_2$ contra $H_0: \pi_1 \neq \pi_2$.
Exemplo 6.5.8. do livro Bolfarine e Sandoval (2001).
\end{exer}


\begin{exer} \rm
% \textit{(Equilíbrio de Hardy-Weinberg)} Seja $\boldsymbol{X} = (X_1, \ldots, X_n)$ uma a. a. de $X \sim Multinomial(N, \pi_1, \pi_2, \pi_3)$. Use o TRV para testar $H_0: \pi_1 = \pi_2 = \pi_3$.  
% COmpare o valor $p$ do TRV com o teste qui quadrado de ajustamento.
Exemplo 10.3.4 do livro Casella e Berger (2002) e 

Exemplo 6.5.9. do livro Bolfarine e Sandoval (2001).
\end{exer}


\begin{exer} \rm
% \textit{(Tabelas $r \times c$)} Suponha que temos uma tabela de contingência $r \times c$ com $n$ indivíduos independentemente selecionados, sendo $n_{ij}$ o número de unidades classificadas na linha $i$ e na coluna $j$, para todo $i=1, \ldots, r$ e $j=1, \ldots, c$. Seja $\pi_{ij}$ a probabilidade de um indivíduo ser classificado na linha $i$ e coluna $j$, tal que $\pi_{ij} \geq 0$ e $\sum_{i=1}^{r} \sum_{j=1}^{c} \pi_{ij} = 1$.
% \begin{enumerate}[a)]
  % \item Encontre o TRV para testar $H_0: \pi_{ij} = a_i b_j$, para algum $a_i > 0$ e $b_j>0$ tais que $\sum_{i=1}^r a_i = 1$ e $\sum_{j=1}^c b_j = 1$, contra a alternativa $H_1: \pi_{ij} \neq a_i b_j$ para pelo menos um par $(i,j)$. 
  % \item Compare o teste do ítem (a) com o teste qui quadrado para independência, para tesar se a variável da linha e da coluna são independentes.
% \end{enumerate}

\end{exer}


\begin{exer} \rm
% \textit{Teste Exato de Fisher}
Exercício 8.48 do livro Casella e Berger (2002)
% <!-- https://newonlinecourses.science.psu.edu/stat504/node/89/ -->

% \textit{(Tabela $2 \times 2$ restrita)} Seja $S_1 \sim Binomial(n_1, \pi_1)$ independente de $S_2 \sim Binomial(n_2, \pi_2)$. Para testar as hipóteses $H_0: \pi_1 = \pi_2$ contra $H_1: \pi_1 > \pi_2$:
% \begin{enumerate}[a)]
%   \item Mostre que sob $H_0$ temos que $S = S_1 + S_2$ é estatística suficiente e $S_1 \vert S = s \sim \textit{Hipergeométrica}(n_1+n_2, n_1, s)$.
%   \item Calcule o valor $p$ (condicional) para o teste exato de Fisher.  
%   \item Compare com o valor $p$ do TRV assintótico.
%   \item Compare com os valores $p$ do TRV e do teste qui quadrado do exercício 5.
% \end{enumerate}
\end{exer}


% \clearpage
\begin{exer} \rm
Exercicio 6.10 do livro Bolfarine e Sandoval (2001). 
% Seja $\boldsymbol{X} = (X_1, \ldots, X_n)$ uma a. a. de $X \sim Normal(\mu_X, \sigma^2_X)$ e $\boldsymbol{Y} = (Y_1, \ldots, Y_m)$ uma a.a. de $Y \sim Normal(\mu_Y, \sigma^2_Y)$, tal que $\boldsymbol{X}$ e $\boldsymbol{Y}$ são independentes.
% \begin{enumerate}[a)]
%   \item Encontre o TRV para testar $H_0: \mu_X = \mu_Y$ contra $H_1: \mu_X \neq \mu_Y$ assumindo que $\sigma^2_X$ e $\sigma^2_Y$ são conhecidos;
%   \item Sendo as variâncias conhecidas, $\sigma^2_X=9$ e $\sigma^2_Y=25$, foi observado $n=9$, $\sum x_i = 3,4$, $m=16$ e $\sum y_i = 4,3$. Qual sua conclusão ao nível de significância 5\%? 
% \end{enumerate}
\end{exer}

% exercício ANOVA, LRT para comparar médias de grupos de variáveis normais???
% Mood, Graybil and Boes do this

\begin{exer} \rm
% Considere variações dos exercícios acima para outras distribuições como $Poisson(\lambda)$, $Exponencial(\lambda)$, $Gama(\alpha, \beta)$, $Beta(\alpha, \beta)$, $Uniforme(0, \theta)$, ...
\end{exer}

\end{document}