\documentclass[letter,11pt]{article}

\usepackage{amsfonts}
\usepackage{amsmath}
\usepackage{amssymb}
\usepackage[brazilian]{babel}
\usepackage{enumerate}
\usepackage[T1]{fontenc}
%\usepackage[ansinew,latin1]{inputenc}
\usepackage[utf8x]{inputenc}
\usepackage{multicol}
\setlength\columnseprule{0.5pt}

\newtheorem{exer}{Exercício}
\newtheorem{teo}{Teorema}

\newcommand{\var}{Var}
\newcommand{\E}{\mathbb{E}}

\newcommand{\mat}[1]{\mbox{\boldmath{$#1$}}}

\usepackage[letterpaper,top=3cm, bottom=2cm, left=2.5cm, right=2.5cm]{geometry}

\begin{document}

%\thispagestyle{empty}
\begin{center}{ \Large MAT02026 - Inferência B }\end{center}

\begin{center}
{\large  \sc Gabarito Lista 6 - TH e IC Bayesianos}
\end{center}
\vspace{5mm}

%%%%%%%%%%%%%%%%%%%%%%%%%%%%%%%%%%%%%%%%%%%%%%%%%%%%%%%%%%%%%%%%%%%%%%%%%%%%%%%
\begin{exer} \rm
Considere que $X_1, \ldots, X_n$ é uma amostra aleatória de uma população com distribuição Binomial Negativa $(r, \theta)$, $r>0$ e $0<\theta<1$, cuja função de probabilidade é dada por:

$$
p(x/ r, \theta) = {x+r-1 \choose x}\theta^r(1-\theta)^x
$$

Assuma que $r$ é conhecido

\begin{enumerate}[a)]
  \item Usando a distribuição conjugada natural de $\theta$, obtenha a distribuição a \textit{posteriori} de $\theta$.
  \item Suponha que foi selecionada uma amostra aleatória dessa população e que obteve $\sum x_i=70$, sendo $r=5$ e $n=10$. Considere as hipóteses:
$H_0: \theta \leq 0.5$ vs $H_1: \theta > 0.5$
Qual das hipóteses apresenta uma maior probabilidade a \textit{posteriori}, admitindo que possui uma informação ``moderada" acerca de $\theta$, a qual é introduzida no modelo por meio de uma distribuição Beta (2,2)?
  \item Indique o valor do fator de bayes em favor de $H_1$. Poderá concluir que existe forte evidência em favor desta hipótese?
\end{enumerate}
\end{exer}


\begin{exer} \rm
Dezesseis consumidores habituais de comida de uma cadeia de fast food foram recrutados
para participar de um estudo cujo objectivo era comparar o sabor de dois tipos de recheio usado num certo tipo de bolo confeccionado com carne de vaca. Um
dos conjuntos de dezasseis recheios que foi analisado tinha sido congelado 8 meses
atrás e mantido num congelador de elevada qualidade (com alterações de temperatura de menos do que um grau na escala de Fahrenheit) durante todo o período de
tempo. Os restantes 16 recheios foram armazenados em congeladores de qualidade média, com variações de temperatura entre 0 e 15 graus Fahrenheit. 

Os gestores da
cadeia de fast food pretendem verificar se a qualidade do refrigeramento altera o sabor
dos recheios e portanto avaliar se valerá a pena o elevado custo associado a uma maior qualidade de refrigeração.

Os produtos são descongelados e cozinhados por um único chef. O planejamento do experimento é feito de forma a que seja “duplamente-cego”, i.e., nem os empregados
que servem o recheio nem os consumidores têm qualquer conhecimento sobre
os produtos que estão a servir (consumir) em qualquer momento. No fim da experiência
observa-se que 13 dos 16 consumidores preferiram os recheios que tinham
sido congelados no congelador de maior qualidade.

\begin{enumerate}[a)]
    \item Qual é o modelo que você sugere para analisar estes dados? Que pressupostos deve impor?
    \item Seja $\theta$ a probabilidade de que os consumidores prefiram o produto mais caro. Sejam Beta(0.5, 0.5), Beta(1.0, 1.0) e Beta(2.0, 2.0) três distribuições que refletem diferentes níveis de conhecimento, a priori, relativamente a $\theta$. Obtenha as distribuições a posteriori correspondentes. Para cada uma delas,
    \begin{itemize}
        \item calcule a média, moda e mediana de $\theta$ a posteriori. Comente os resultados que obtiver.
        \item  Indique a $P(\theta > 0.6 | x)$.
        \item Calcule o intervalo HPD de 95\% de credibilidade para $\theta$.
        \item Calcule o fator de Bayes para $H_0 : \theta \geq 0.6$ vs. $H_1 : \theta < 0.6$. Faça os comentários que considerar adequados.
    \end{itemize}
\end{enumerate}
\end{exer}


\begin{exer} \rm 
Sejam $X_1, X_2,\ldots, X_n$ variáveis aleatórias i.i.d. com distribuição de Poisson($\theta$), $\theta> 0$. Considere a distribuição Gama(1, 1) como distribuição a priori para $\theta$. Obtenha o intervalo de credibilidade de 90\% de igual probabilidade e o intervalo de
credibilidade HPD de 90\% para $\theta$. Considerando uma amostra aleatória
de dimensão $n = 10$ para a $\sum x_i = 6$.
\end{exer}


\begin{exer} \rm 
Numa certa população, seja $\theta$ a proporção de indivíduos que têm uma determinada doença. Suponha que se pretende testar a hipótese $H_0 : \theta = 0.2$ contra a hipótese alternativa $H_1 : \theta = 0.3$. Informações anteriores mostram que $P(H_0) = 0.25$. Suponha que se observam $n$ indivíduos verificando-se que $x$ apresentam a doença.
Calcule o fator de Bayes a favor de $H_0$. Para que valores de $x$ é que se tem $P(H_0 | x) > P(H_1 | x)$?
\end{exer}


\begin{exer} \rm
Ler o material do blog https://www.countbayesie.com/blog/2016/3/16/bayesian-reasoning-in-the-twilight-zone
\end{exer}


\begin{exer} \rm
%13 vanessa
Deseja-se estimar $\theta$: a proporção de residentes de determinada cidade que concordam com a construção de um presídio na cidade. Para isto, observou-se uma amostra de 100 pessoas, das quais 26 concordavam com a construção do presídio.

\begin{enumerate}[a)]
  \item Antes de observar a amostra, um especialista afirmou que essa proporção se comportava conforme uma distribuição Beta com esperança a priori de 0,20 e variância a \textit{priori} de 0,0064. Encontre os parâmetros desta priori e construa a \textit{posteriori}.
  \item Calcule a estimativa de MVG e de Bayes.
  \item  Através do R, encontre o ICs 95\% Central e HPD.
  \item Seja $H_0 : \theta \leq  0, 5$. Qual a probabilidade desta hipótese ser verdadeira, a \textit{posteriori}?
\end{enumerate}
\end{exer}


\begin{exer} \rm
%14 vanessa
Deseja-se estimar o diâmetro médio de determinada peça produzida em certa fábrica. A experiência indica que esse diâmetro é normalmente distribuído com variância 4$cm^2$.
Um engenheiro escolheu como priori para $\mu$ uma $N(\mu_0 = 30; \sigma_0^2=16)$, pois acredita ser praticamente impossível que esse diâmetro seja menor que 18 cm ou maior que 42 cm. Uma amostra de 12 peças foi observada e se obteve $\overline{x}=32$cm.

\begin{enumerate}[a)]
  \item Calcule o IC 95\% HPD para $\mu$. Calcule também o 1º e 3º quartis.
  \item Usando priori $\pi(\mu)$ propocional a 1, repita o item a.
  \item Qual seria o IC 95\% encontrado pela inferência clássica? Compare com os itens anteriores.
\end{enumerate}
\end{exer}


\begin{exer} \rm
% lista6_Marcio exercicio 7
Seja $X$ o tempo de vida de uma lâmpada (em mil horas) fabricada por a certa companhia. Considera-se que $X$ é  uma variável aleatóia com densidade
\[f(x|\theta)=\theta e^{-\theta x},\quad x>0.\]
Considere uma priori para $\theta$: 
\[\pi(\theta)=16\theta e^{-4\theta},\quad \theta>0.\]

\begin{enumerate}[a)]
    \item Encontre a distribuição a posteriori para $\theta$.
    \item Encontre o estimador de Bayes de E($X$) e Var($X$). 
\end{enumerate}
\end{exer}


\end{document}